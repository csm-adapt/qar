\documentclass{article}

\usepackage[utf8]{inputenc}
\usepackage[margin=1in]{geometry}
\usepackage{amsfonts}
\usepackage{amsmath}
\usepackage{amsthm}
\usepackage{booktabs}
\usepackage{mathtools}

\newtheorem{theorem}{Theorem}
\newtheorem{corollary}[theorem]{Corollary}

\setlength{\parindent}{0pt}

\title{On the Composition of Fourier and Power Series}
\date{\today}

\begin{document}
    \maketitle

    \section{Power Variable under Transformation of a Fourier Function}

    \subsection{Notation}
    Let $x \in \mathbb{R}$ be a scalar. Let $y \in \mathbb{R}$ be a scalar. Let $\sigma \in \mathcal{C}(\mathbb{R})$ be an Fourier function.\\
    
    Then, $\sigma$ can be represented as
    \begin{equation}
        \sigma(x) = \sum_{n \in \mathbb{Z}} s_n e^{2\pi inx}
        \label{eqn1:sigma-fourier}
    \end{equation}
    
    Suppose that $x$ and $y$ share the relation
    \begin{equation}
        y = \sigma(x).
        \label{eqn1:xy-relation}
    \end{equation}
    
    Suppose that $x$ can be represented as a Fourier series of $z$, another scalar.
    \begin{equation}
        x = \sum_{k=0}^{\infty} a_k z^k
        \label{eqn1:x-power}
    \end{equation}
    
    Suppose that wish to find a similar Fourier representation for $y$ in terms of $z$.
    \begin{equation}
        y = \sum_{k=0}^{\infty} b_k z^k
        \label{eqn1:y-power}
    \end{equation}
    
    \subsection{Expansion}
    We expand the relation given in Equation (\ref{eqn1:xy-relation}) by using Equations (\ref{eqn1:sigma-fourier}) and (\ref{eqn1:x-power}).
    
    \begin{align}
        y &= \sigma(x) \nonumber \\
          &= \sum_{n\in\mathbb{Z}} s_n e^{2\pi inx} \nonumber \\
          &= \sum_{n\in\mathbb{Z}} s_n e^{2\pi in\sum_{k=0}^{\infty} a_k z^k} \nonumber \\
          &= \sum_{n\in\mathbb{Z}} s_n \sum_{q=0}^{\infty} \frac{\left(2\pi in\sum_{k=0}^{\infty} a_k z^k\right)^q}{q!} \nonumber \\
          &= \sum_{n\in\mathbb{Z}} s_n \sum_{q=0}^{\infty} \frac{(2\pi in)^q}{q!} \left(\sum_{k=0}^{\infty}a_k z^k\right)^q \nonumber \\
          &= \sum_{n\in\mathbb{Z}} s_n \sum_{q=0}^{\infty} \frac{(2\pi in)^q}{q!} \left(\sum_{q_0 + q_1 + \cdots = q} \binom{q}{q_0, q_1, \ldots} \prod_{k=0}^{\infty} \left(a_k z^k\right)^{q_k} \right) \nonumber \\
          &= \sum_{n\in\mathbb{Z}} s_n \sum_{q=0}^{\infty} \frac{(2\pi in)^q}{q!} \left(\sum_{q_0 + q_1 + \cdots = q} \binom{q}{q_0, q_1, \ldots} \prod_{k=0}^{\infty} a_k^{q_k} z^{q_k k} \right)
          \label{eqn1:xy-expansion}
    \end{align}
    
    \subsection{Coefficient Extraction}
    
    In order to find the coefficients $b_i$ for Equation (\ref{eqn1:y-power}), we must extract the index and power constraints from Equation (\ref{eqn1:xy-expansion}).\\
    
    The index constraint is
    
    \begin{equation}
        q = \sum_{m=0}^{\infty} q_m.
    \end{equation}
    
    The power constraint is
    
    \begin{equation}
        i = \sum_{m=0}^{\infty} m q_m.
    \end{equation}
    
    Therefore, the expression for the coefficient $g_i$ is
    
    \begin{equation}
        b_i = \sum_{n=0}^{\infty} s_n \sum_{q}^{\infty} \frac{(2\pi in)^q}{q!} \sum_{\substack{q_0 + q_1 + \cdots q_i = q \\ 0q_0 + 1q_1 + \cdots iq_i = i}} \binom{q}{q_0, q_1, \ldots, q_i} \prod_{k=0}^{\infty} a_k^{q_k}.
        \label{eqn1:icoeff}
    \end{equation}

    \subsection{Analysis of Sine Transformation}

    We can let the transformation $\sigma(x) = \sin(x)$. We then have

    \begin{equation}
        \sigma(x) = \sin(x) = \frac{i}{2} e^{-ix} - \frac{i}{2} e^{ix}
    \end{equation}

    so that $s_{-1} = i / 2$, $s_{1} = -i / 2$, and $s_{k} = 0$ for $k \neq \pm 1$. Inserting these coefficients into Equation (\ref{eqn1:icoeff}), we get

    \begin{align}
        b_j
        &= \sum_{n \in \{-1, 1\}}^{\infty} s_n \sum_{q=0}^{\infty} \frac{(2\pi in)^q}{q!} \sum_{\substack{q_0 + q_1 + \cdots q_j = q \\ 0q_0 + 1q_1 + \cdots jq_j = j}} \binom{q}{q_0, q_1, \ldots, q_j} \prod_{k=0}^{\infty} a_k^{q_k} \nonumber \\
        &= \frac{i}{2} \left[ \sum_{q=0}^{\infty} \frac{(-1)^q(2\pi i)^q}{q!} \sum_{\substack{q_0 + q_1 + \cdots q_i = q \\ 0q_0 + 1q_1 + \cdots jq_j = j}} \binom{q}{q_0, q_1, \ldots, q_j} \prod_{k=0}^{\infty} a_k^{q_k}\right] - \frac{i}{2} \left[ \sum_{q=0}^{\infty} \frac{(2\pi i)^q}{q!} \sum_{\substack{q_0 + q_1 + \cdots q_j = q \\ 0q_0 + 1q_1 + \cdots jq_j = j}} \binom{q}{q_0, q_1, \ldots, q_j} \prod_{k=0}^{\infty} a_k^{q_k}\right] \nonumber \\
        &= \frac{i}{2} \sum_{q=0}^{\infty} \left[\frac{(2\pi i)^q}{q!}\left((-1)^q - 1\right) \sum_{\substack{q_0 + q_1 + \cdots q_j = q \\ 0q_0 + 1q_1 + \cdots jq_j = j}} \binom{q}{q_0, q_1, \ldots, q_j} \prod_{k=0}^{\infty} a_k^{q_k}\right] \nonumber \\
        &= -i \sum_{q=0}^{\infty} \left[\frac{(2\pi i)^{2q + 1}}{(2q+1)!} \sum_{\substack{q_0 + q_1 + \cdots q_j = 2q + 1 \\ 0q_0 + 1q_1 + \cdots jq_j = j}} \binom{2q + 1}{q_0, q_1, \ldots, q_j} \prod_{k=0}^{\infty} a_k^{q_k}\right] \nonumber \\
        &= \sum_{q=0}^{\infty} \left[\frac{(-1)^{q}(2\pi)^{2q + 1}}{(2q+1)!} \sum_{\substack{q_0 + q_1 + \cdots q_j = 2q + 1 \\ 0q_0 + 1q_1 + \cdots jq_j = j}} \binom{2q + 1}{q_0, q_1, \ldots, q_j} \prod_{k=0}^{\infty} a_k^{q_k}\right]
    \end{align}

    This can also be found using a similar expansion as in Equation (\ref{eqn:xy-expansion}) with $y = sin(x)$. We will use this expansion to prove the following theorem about the instability of the Taylor expansion of the iterated sine function.

    \begin{theorem}
        Let $\sin^{(k)}(x)$ be the iterated sine function. That is $\sin^{(1)}(x) = \sin(x)$, $\sin^{(2)}(x) = \sin(\sin(x))$, $\sin^{(3)}(x) = \sin(\sin(\sin(x)))$, etc. Then, each iterated sine function $\sin^{(k)}(x)$ has a Maclaurin Series representation
        
        \begin{equation}
            \sin^{(k)}(x) = s_0^{(k)} + s_1^{(k)} x + s_2^{(k)} x^2 + \cdots
        \end{equation}
        
        and $\left|s_i^{(m)}\right| \geq \left|s_i^{(n)}\right|$ if $m \geq n$ for all $i \in \mathbb{N}$.
    \end{theorem}
    \begin{proof}
        We will prove this theorem by strong induction on $m$.\\
        
        First, we will start with $m = 1$. Then, either $n = 0$ or $n = 1$.
        
        \begin{enumerate}
            \item When $n = 0$, $\sin^{(n)}(x) = id(x)$ is the identity of $x$ which has $s_1^{(0)} = 1$ and $s_i^{(0)} = 0$ for $i \neq 1$. The Maclaurin Series expansion of $\sin(x)$ has $s_{2n}^{(1)} = 0$ and $s_{2n+1}^{(1)} = \frac{(-1)^n}{(2n+1)!}$. Since, $s_{1}^{(1)} = \frac{(-1)^0}{1!} = 1$, then $\left|s_{i}^{(1)}\right| \geq \left|s_{i}^{(0)}\right|,\ \forall i \in \mathbb{N}$. 
            
            \item When $n = 1 = m$, we trivially obtain $\left|s_{i}^{(m)}\right| \geq \left|s_{i}^{(n)}\right|,\ \forall i \in \mathbb{N}$.
        \end{enumerate}
        
        Second, we will prove that the theorem holds for $m = k$ under the assumption that the theorem holds for $m < k$ for $k > 1$. In this case, either $n < m$ or $n = m$.
        
        \begin{enumerate}
            \item When $n < m$, then, by assumption, we have $\left|s_{i}^{(m-1)}\right| \geq \left|s_{i}^{(n)}\right|,\ \forall i \in \mathbb{N}$ and we need only show that $\left|s_{i}^{(m)}\right| \geq \left|s_{i}^{(m-1)}\right|,\ \forall i \in \mathbb{N}$. [PROOF NOT COMPLETE HERE]
            
            \item When $n = m$, we trivially obtain $\left|s_{i}^{(m)}\right| \geq \left|s_{i}^{(n)}\right|,\ \forall i \in \mathbb{N}$.
        \end{enumerate}
        
        Thus, we have $\left|s_i^{(m)}\right| \geq \left|s_i^{(n)}\right|$ if $m \geq n$ for all $i \in \mathbb{N}$ so the theorem is proven.
    \end{proof}

    For instance, $\sin^{(k)}$ has the following first $10$ coefficients for $k = 0, 1, \ldots, 7$.

    \begin{table}[h]
    \centering
    \begin{tabular}{@{}l|cccccccccc@{}}
    $k$ & $c_{0}^{(k)}$ & $c_{1}^{(k)}$ & $c_{2}^{(k)}$ & $c_{3}^{(k)}$ & $c_{4}^{(k)}$ & $c_{5}^{(k)}$ & $c_{6}^{(k)}$ & $c_{7}^{(k)}$ & $c_{8}^{(k)}$ & $c_{9}^{(k)}$ \\ \midrule
    $0$ & $0$ & $1$ & $0$ & $0$    & $0$ & $0$       & $0$ & $0$           & $0$ & $0$              \\
    $1$ & $0$ & $1$ & $0$ & $-1/6$ & $0$ & $1/120$   & $0$ & $-1/5040$     & $0$ & $1/362880$       \\
    $2$ & $0$ & $1$ & $0$ & $-2/6$ & $0$ & $12/120$  & $0$ & $-128/5040$   & $0$ & $1872/362880$    \\
    $3$ & $0$ & $1$ & $0$ & $-3/6$ & $0$ & $33/120$  & $0$ & $-731/5040$   & $0$ & $25857/362880$   \\
    $4$ & $0$ & $1$ & $0$ & $-4/6$ & $0$ & $64/120$  & $0$ & $-2160/5040$  & $0$ & $121600/362880$  \\
    $5$ & $0$ & $1$ & $0$ & $-5/6$ & $0$ & $105/120$ & $0$ & $-4765/5040$  & $0$ & $368145/362880$  \\
    $6$ & $0$ & $1$ & $0$ & $-6/6$ & $0$ & $156/120$ & $0$ & $-8896/5040$  & $0$ & $873936/362880$  \\
    $7$ & $0$ & $1$ & $0$ & $-7/6$ & $0$ & $217/120$ & $0$ & $-14903/5040$ & $0$ & $1776817/362880$ \\
    \end{tabular}
    \end{table}

    \begin{theorem}
        The sequence of values $s_{2n+1}^{(k)}$ as a function of $k$ is a polynomial of degree $n$.
    \end{theorem}
    \begin{proof}
        [PROOF NOT COMPLETE HERE]
    \end{proof}

    Therefore, not only are the coefficients increasing, but higher-order coefficients grow more quickly than lower-order coefficients.

    \begin{corollary}
        Conjecture: For $k > 1$, the radius of convergence of the power series of $\sin^{(k)}(x)$ is $\rho = \pi^{-^{k}2}$ about $x = 0$.
    \end{corollary}
    \begin{proof}
        [PROOF NOT COMPLETE HERE]
    \end{proof}

    Thus, for $k > 1$, the power series of $\sin^{(k)}(x)$ has a finite radius of convergence contrary to the infinite radius of convergece of the power series of $\sin(x)$.

    \begin{theorem}
        The numbers $\left\{5! \cdot s_{5}^{(k)}\right\}_{k=0}^{\infty}$ form the sequence of dodecagonal numbers.
    \end{theorem}
    \begin{proof}
        [PROOF NOT COMPLETE HERE]
    \end{proof}

    \section{Fourier Variable under Transformation of an Power Function}

    \subsection{Notation}
    Let $x \in \mathbb{R}$ be a scalar. Let $y \in \mathbb{R}$ be a scalar. Let $\sigma \in \mathcal{C}(\mathbb{R})$ be a function with a power representation.\\

    Then, $\sigma$ can be represented as
    \begin{equation}
        \sigma(x) = \sum_{n=0}^{\infty} s_n x^n.
        \label{eqn2:sigma-analytic}
    \end{equation}

    Suppose that $x$ and $y$ share the relation
    \begin{equation}
        y = \sigma(x).
        \label{eqn2:xy-relation}
    \end{equation}

    Suppose that $x$ can be represented as a Fourier series of $z$, another scalar.
    \begin{equation}
        x = \sum_{k \in \mathbb{Z}} f_k e^{2\pi i k z}
        \label{eqn2:x-fourier}
    \end{equation}

    Suppose that wish to find a similar Fourier representation for $y$ in terms of $z$.
    \begin{equation}
        y = \sum_{k \in \mathbb{Z}} g_k e^{2\pi i k z}
        \label{eqn2:y-fourier}
    \end{equation}

    \subsection{Expansion}
    We expand the relation given in Equation (\ref{eqn2:xy-relation}) by using Equations (\ref{eqn2:sigma-analytic}) and (\ref{eqn2:x-fourier}).

    \begin{align}
        y &= \sigma(x) \nonumber \\
        &= \sum_{n=0}^{\infty} s_n x^n \nonumber \\
        &= \sum_{n=0}^{\infty} s_n \left(\sum_{k \in \mathbb{Z}} f_k e^{2\pi i k z}\right)^n \nonumber \\
        &= \sum_{n=0}^{\infty} s_n \left(\sum_{\cdots k_{-1} + k_{0} + k_{1} + \cdots = n} \binom{n}{\ldots, k_{-1}, k_{0}, k_{1}, \ldots} \prod_{t \in \mathbb{Z}} \left(f_t e^{2\pi i t z}\right)^{k_t} \right) \nonumber \\
        &= \sum_{n=0}^{\infty} s_n \left(\sum_{\cdots k_{-1} + k_{0} + k_{1} + \cdots = n} \binom{n}{\ldots, k_{-1}, k_{0}, k_{1}, \ldots} \prod_{t \in \mathbb{Z}} f_t^{k_t} e^{2\pi i t k_t z} \right)
        \label{eqn2:xy-expansion}
    \end{align}

    \subsection{Coefficient Extraction}

    In order to find the coefficients $g_i$ for Equation (\ref{eqn2:y-fourier}), we must recognize an index constraint and a power constraint on the term $e^{2\pi i t k_t z}$ from Equation (\ref{eqn2:xy-expansion}).\\

    The index constraint is

    \begin{equation}
        n = \sum_{m \in \mathbb{Z}} k_m.
    \end{equation}

    The power constraint is

    \begin{equation}
        i = \sum_{t \in \mathbb{Z}} tk_t
    \end{equation}

    Therefore, the expression for the coefficient $g_i$ is

    \begin{equation}
        g_i = \sum_{n=0}^{\infty} s_n \left(\sum_{\substack{n = \sum_{m \in \mathbb{Z}} k_m \\ i=\sum_{t \in \mathbb{Z}} tk_t}} \binom{n}{\ldots, k_{-1}, k_{0}, k_{1}, \dots} \prod_{t \in \mathbb{Z}}f_t^{k_t} \right).
    \end{equation}

    This cannot be simplified as in the variable power series case since $i = \sum_{t \in \mathbb{Z}} tk_t$ cannot be reduced. All modes must be considered.

    \section{Fourier Variable under Transformation of an Fourier Function}

    \subsection{Notation}
    Let $x \in \mathbb{R}$ be a scalar. Let $y \in \mathbb{R}$ be a scalar. Let $\sigma \in \mathcal{C}(\mathbb{R})$ be a function with a Fourier representation.\\

    Then, $\sigma$ can be represented as
    \begin{equation}
        \sigma(x) = \sum_{n \in \mathbb{Z}} s_n e^{2 \pi i n x}.
        \label{eqn3:sigma-fourier}
    \end{equation}

    Suppose that $x$ and $y$ share the relation
    \begin{equation}
        y = \sigma(x).
        \label{eqn3:xy-relation}
    \end{equation}

    Suppose that $x$ can be represented as a Fourier series of $z$, another scalar.
    \begin{equation}
        x = \sum_{k \in \mathbb{Z}} f_k e^{2\pi i k z}
        \label{eqn3:x-fourier}
    \end{equation}

    Suppose that wish to find a similar Fourier representation for $y$ in terms of $z$.
    \begin{equation}
        y = \sum_{k \in \mathbb{Z}} g_k e^{2\pi i k z}
        \label{eqn3:y-fourier}
    \end{equation}

    \subsection{Expansion}
    We expand the relation given in Equation (\ref{eqn3:xy-relation}) by using Equations (\ref{eqn3:sigma-fourier}) and (\ref{eqn3:x-fourier}).

    \begin{align}
        y &= \sigma(x) \nonumber \\
        &= \sum_{n \in \mathbb{Z}} s_n e^{2 \pi i n x} \nonumber \\
        &= \sum_{n \in \mathbb{Z}} s_n e^{2 \pi i n \left(\sum_{k \in \mathbb{Z}} f_k e^{2\pi i k z}\right)} \nonumber \\
        &= \sum_{n \in \mathbb{Z}} s_n \sum_{q=0}^{\infty} \frac{\left(2 \pi i n \left[\sum_{k \in \mathbb{Z}} f_k e^{2\pi i k z}\right]\right)^q}{q!} \nonumber \\
        &= \sum_{n \in \mathbb{Z}} s_n \sum_{q=0}^{\infty} \frac{\left(2\pi in\right)^q}{q!} \left(\sum_{k \in \mathbb{Z}} f_k e^{2\pi ikz} \right)^q \nonumber \\
        &= \sum_{n \in \mathbb{Z}} s_n \sum_{q=0}^{\infty} \frac{(2\pi i n)^q}{q!} \sum_{\cdots + q_{-1} + q_{0} + q_{1} + \cdots = q} \binom{q}{\ldots, q_{-1}, q_{0}, q_{1}, \ldots} \prod_{m \in \mathbb{Z}} (f_m e^{2\pi imz})^{q_m} \nonumber \\
        &= \sum_{n \in \mathbb{Z}} s_n \sum_{q=0}^{\infty} \frac{(2\pi i n)^q}{q!} \sum_{\cdots + q_{-1} + q_{0} + q_{1} + \cdots = q} \binom{q}{\ldots, q_{-1}, q_{0}, q_{1}, \ldots} \prod_{m \in \mathbb{Z}} f_m^{q_m} e^{2\pi q_m imz}
        \label{eqn3:xy-expansion}
    \end{align}

    \subsection{Coefficient Extraction}

    In order to find the coefficients $g_i$ for Equation (\ref{eqn3:y-fourier}), we must recognize an index constraint and a power constraint on the term $e^{2\pi q_m imz}$ from Equation (\ref{eqn3:xy-expansion}).\\

    The index constraint is

    \begin{equation}
        q = \sum_{m \in \mathbb{Z}} q_m.
    \end{equation}

    The power constraint is

    \begin{equation}
        i = \sum_{t \in \mathbb{Z}} m q_m
    \end{equation}

    Therefore, the expression for the coefficient $g_i$ is

    \begin{equation}
        g_i = \sum_{n=0}^{\infty} s_n \sum_{q=0}^{\infty} \frac{(2\pi in)^q}{q!} \left(\sum_{\substack{q = \sum_{m \in \mathbb{Z}} q_m \\ i=\sum_{m \in \mathbb{Z}} m q_m}} \binom{n}{\ldots, q_{-1}, q_{0}, q_{1}, \dots} \prod_{m \in \mathbb{Z}}f_t^{q_m} \right).
    \end{equation}

    This cannot be simplified as in the variable power series case since $i = \sum_{m \in \mathbb{Z}} m q_m$ cannot be reduced. All modes must be considered.

\end{document}