\section{Methods}\label{methods}

Fully dense neural network (NN) architectures, such as the one shown in Figure~\ref{fig:nn-1}, perform a sequence of affine transformations, ${\bold z}_i \leftarrow \boldsymbol\theta_i {\bold x}^{(i)}$, followed by element-wise functional operations, $\sigma({\bold z}_i)$ to introduce non-linearity at each layer; that is, each layer stretches and distorts the underlying space.
\begin{figure}[htbp]
\begin{center}
\includegraphics[width=0.4\textwidth]{fig/neural-network-01}
\caption{Schematic view of a fully dense neural network. Each sequence of affine and non-linear transformations are captured in the function, $f_i({\bold x}): {\bold x}^{(i+1)} \leftarrow \sigma(\boldsymbol\theta_i {\bold x}^{(i)})$}
\label{fig:nn-1}
\end{center}
\end{figure}

The resulting network,
\begin{equation}
	f(x) = \sigma(\boldsymbol\theta_n \sigma(\boldsymbol\theta_{n-1} \sigma(\ldots \boldsymbol\theta_2 \sigma(\boldsymbol\theta_1{\bold x}))))
	\label{eqn:nn analytical form}
\end{equation}
is an arbitrary function generator, but at present, the network weights $\boldsymbol\theta_i$ can not map back to analytic forms that capture and describe the underlying physics. There are, however, many such mappings through polynomial series expansions,
\begin{equation}
	f(x) = \sum_{n=0}^\infty a_n x^n
\end{equation}

We hypothesize that the physics of a process can be extracted by fitting the polynomial expansions of known physical relationships to the polynomial coefficients of a polynomial series expansion of Equation~(\ref{eqn:nn analytical form}).

Although ReLU (rectified linear units) have become a more common activation function, its discontinuity at $x = 0$ requires an infinite series to fully capture the behavior at this transition. However, the softplus function,%However, the sigmoid function,
%\begin{equation}
%	\sigma(x) = \frac{1}{1 + e^{-x}}
%	\label{eqn:sigmoid}
%\end{equation}
%is a special case of the generating function for the Euler polynomial coefficients,
%\begin{equation}
%	\frac{2e^{x t}}{1 + e^t} = \sum_{n=0}^\infty E_n(x) \frac{t^n}{n!}
%\end{equation}
%where, for $x = 0$,
%\begin{equation}
%	\sigma(x) = \frac{1}{2} \sum_{n=0}^\infty E_n(0) \frac{(-1)^n}{n!}.
%	\label{eqn:sigmoid Euler expansion}
%\end{equation}
%
%The Euler polynomials at $x=0$,
%\begin{equation}
%	E_n(0) = -2(n+1)^{-1} \left( 2^{n+1} - 1 \right) B_{n+1}
%\end{equation}
%where $B_n$ is the $n^\textrm{th}$ Bernoulli number. Since Bernoulli numbers of odd index, with the exception of $B_1$, are zero, $E_i(0) = 0$ for $i = 2, 4, 6, \ldots, 2n$. Therefore, the summand and limits of Equation~(\ref{eqn:sigmoid Euler expansion}) change to
%\begin{equation}
%	\sigma(x) = \frac{1}{2} - \frac{1}{2} \sum_{n=1}^\infty \left( \frac{E_{2n-1(0)}}{(2n-1)!} \right) x^{2n-1}.
%\end{equation}
%
%The series representation of $E_{2n-1}(x)$
%\begin{equation}
%	E_{2n-1}(x) = \frac{(-1)^n 4 (2n - 1)!}{\pi^{2n+1}} \sum_{k=0}^\infty \frac{\cos [(2k + 1) \pi x]}{(2k + 1)^{2n}}
%\end{equation}
%such that,
%\begin{equation}
%	E_{2n-1}(0) = \frac{(-1)^n 4 (2n - 1)!}{\pi^{2n+1}} \sum_{k=0}^\infty \frac{1}{(2k + 1)^{2n}}
%\end{equation}
%and therefore,
%\begin{eqnarray}
%	\sigma(x) & = & \frac{1}{2} - \sum_{n=1}^\infty 2 \frac{(-1)^n}{\pi^{2n}} \left( \sum_{k=0}^\infty \frac{1}{(2k+1)^{2n}} \right) x^{2n-1} \\
%		& = & \frac{1}{2} - \sum_{n=1}^\infty 2 \frac{(-1)^n}{\pi^{2n}} \left( 4^{-n} \left( 4^n - 1 \right) \zeta(2n) \right) x^{2n-1} \nonumber \\
%		& = & \frac{1}{2} - \sum_{n=1}^\infty \underbrace{2 \left( \frac{-1}{4\pi^2} \right)^n \left( 4^n - 1 \right) \zeta(2n)}_{a_n} x^{2n-1} \nonumber \\
%		& = & \sum_{n=0}^\infty a_n x^n,\ a_n = \left\{ \begin{array}{l l}
%			1/2	& n = 0 \\
%			-2 \left( \frac{-1}{4\pi^2} \right)^{(n+1)/2} \left( 4^{(n+1)/2} - 1 \right)\zeta(n+1)	& n\ \text{odd} \\
%			0	& n\ \text{even}
%		\end{array}\right.
%		\label{eqn:sigmoid zeta expansion}
%\end{eqnarray}

\begin{equation}
	f(x) = log(1+e^x)
\end{equation} 

The series expansion for the exponential
\begin{equation}
	e^x = \sum_{k=0}^\infty \frac{x^k}{k!}
\end{equation}
where, $a_k = \frac{1}{k!}$, can be expressed as
\begin{equation}
	e^x = \sum_{k=0}^\infty x^k a_k
\end{equation}

A similar series expansion for the logarithm
\begin{equation}
	log(1+x) = \sum_{n=1}^\infty (-1)^{n+1} \frac{x^n}{n}
\end{equation}
where, $x = e^x$,
allows for the softplus function to be represented as
\begin{equation}
	f(x) = \sum_{n=1}^\infty \frac{(-1)^{n+1}}{n} 				(\sum_{k=0}^\infty x^k a_k)^n
\end{equation}

If the expansion of $e^x$ is performed, then it can be seen that
\begin{equation}
	(\sum_{k=0}^\infty x^k a_k)^n = 								(a_0+x(a_1+x(a_2+x(...))))^n
\end{equation}

Using the binomial theorem
\begin{equation}
	(a+b)^n = \sum_{k=0}^{n} \binom{n}{k} a^{n-k} b^k
\end{equation}
where, $a=a_0$ and $b=(x(a_1+x(a_2+x(...))))^n$, the series expansion of the exponential can be represented as
\begin{equation}
	(\sum_{k=0}^\infty x^k a_k)^n = \sum_{k=0}^{n} 				\binom{n}{k} a_0^{n-k} (a_1+x(a_2+x(...)))^n x^n
\end{equation}

Continuing with this approach, it can be seen that 
	