\documentclass{article}

\usepackage[utf8]{inputenc}
\usepackage[margin=1in]{geometry}
\usepackage{amsfonts}
\usepackage{amsmath}
\usepackage{amsthm}
\usepackage{booktabs}
\usepackage{mathtools}

\newtheorem{theorem}{Theorem}
\newtheorem{lemma}[theorem]{Lemma}
\newtheorem{corollary}[theorem]{Corollary}

\title{On the Truncation Error of Fourier Series}
\date{\today}

\begin{document}

\setlength{\parindent}{0pt}
\maketitle

\section{Notation}

\subsection{Fourier Series}

We will consider the notion of a generalized Fourier series with wave period $L$. We will write the Fourier expansion of a function $f$ in its exponential form,
\begin{equation}
    f(x) = \sum_{n=-\infty}^{\infty} c_n e^{\frac{2\pi}{L}inx}
\end{equation}
where $c_{n} \in \mathbb{C}$ are determined by the series expansion of a function using the following expression.
\begin{equation}
    c_n = \frac{1}{L} \int_{0}^{L} f(x) e^{-\frac{2\pi}{L}inx} dx
\end{equation}

In practical settings, we cannot compute the entirety of a Fourier series. Instead, we compute the truncated Fourier series which does not contain all infinitely many terms. The truncated Fourier series of degree $N$ is denoted as
\begin{equation}
    \hat{f}_N(x) = \sum_{n=-N}^{N} \hat{c}_n e^{\frac{2\pi}{L}inx}
\end{equation}
where $\hat{c}_n$ are the same coefficients as $c_n$ if included in the truncation. That is
\begin{equation}
    \hat{c}_n =
    \begin{cases}
        c_n, &|n| \leq N, \\
        0, &|n| > N
    \end{cases}
\end{equation}

\subsection{Function Norms}

We use function norms to measure the magnitude of a function over a compact support. In general, the $L_p$ norm of an integrable function $f$ is defined by:
\begin{equation}
    \lVert f \rVert_p = \left[\int_{0}^{L} |f(x)|^p dx \right]^{\frac{1}{p}}
\end{equation}

Specifically, we will be using the following function norms throughout this text:
\begin{align}
    \lVert f \rVert_1 &= \int_{0}^{L} |f(x)|\ dx \\
    \lVert f \rVert_2 &= \sqrt{\int_{0}^{L} |f(x)|^2\ dx}
\end{align}

\section{Error Equality via Integration}

We would like to determine the accuracy of a truncated Fourier series function compared to its exact function. To do this, we find the $L^2$ norm of the difference between the two functions.

\begin{lemma}[Parseval's Identity]
    Suppose that $f$ is a square integrable function ($\int_{0}^{L} |f(x)|^2\ dx$ exists) whose Fourier series converges uniformly to $f$. Then,
    \begin{equation}
        \lVert f \rVert_2^2 = \int_{0}^{L} |f(x)|^2\ dx = L \sum_{n=-\infty}^{\infty} |c_n|^2
    \end{equation}
\end{lemma}

\begin{corollary}
    Suppose that $f$ is a square integrable function whose Fourier series converges uniformly to $f$. Then,
    \begin{equation*}
        \lVert f - \hat{f}_N \rVert_2^2 = \lVert f \rVert_2^2 - \lVert \hat{f}_N \rVert_2^2
    \end{equation*}
\end{corollary}
\begin{proof}
    Using Parseval's Identity, we can rewrite the LHS of our equation.
    \begin{align}
        \lVert f - \hat{f}_N \rVert_2^2
        &= L \sum_{n=-\infty}^{\infty} |c_n - \hat{c}_n|^2 \\
        &= L \sum_{n=-\infty}^{\infty} (c_n^2 - 2c_n \hat{c}_n + \hat{c}_n^2) \\
        &= L \sum_{n=-\infty}^{\infty} c_n^2 + L \sum_{n=-\infty}^{\infty} \hat{c}_n^2 - 2L \sum_{n=-\infty}^{\infty} c_n \hat{c}_n \\
        &= \lVert f \rVert_2^2 + \lVert \hat{f}_N \rVert_2^2 - 2L \sum_{n=-\infty}^{\infty} c_n \hat{c}_n
    \end{align}
    We now analyze this final summation term. By substituting the exact expression for $\hat{c}_n$, we find
    \begin{align}
        L \sum_{n=-\infty}^{\infty} c_n \hat{c}_n
        &= L \sum_{n=-\infty}^{\infty} \begin{cases}c_n^2,& |n| \leq N,\\ 0,& |n| > N\end{cases} \\
        &= L \sum_{n=-N}^{N} c_n^2 = L \sum_{n=-N}^{N} \hat{c}_n^2 = L \sum_{n=-\infty}^{\infty} \hat{c}_n^2 \\
        &= \lVert \hat{f}_N \rVert_2^2.
    \end{align}
    Therefore, plugging this result back in, we obtain the RHS of the equation.
    \begin{equation}
        \lVert f - \hat{f}_N \rVert_2^2 = \lVert f \rVert_2^2 + \lVert \hat{f}_N \rVert_2^2 - 2 \lVert \hat{f}_N \rVert_2^2 = \lVert f \rVert_2^2 - \lVert \hat{f}_N \rVert_2^2.
    \end{equation}
    This completes the proof.
\end{proof}

Thus, if we know both $f$ and $f_N$, we can compute the error:
\begin{equation*}
    \int_{0}^{L} |(f - \hat{f}_N)(x)|^2\ dx = \int_{0}^{L} |f(x)|^2\ dx - \int_{0}^{L} |\hat{f}_N(x)|^2\ dx
\end{equation*}

\section{Error Bound via Continuity}

Often, we wish to know the degree of terms $N$ we need to calculate of the truncated Fourier series $\hat{f}_N$ before computing it. For this purpose, we form an inequality on $N$ to achieve some level of accuracy $\varepsilon$.

\begin{lemma}
    Let $f$ be a function that has $p$ continuous derivatives ($f \in \mathcal{C}^p$). Then, in the Fourier series of $f$,
    \begin{equation}
        |c_n| \leq \frac{\lVert f^{(p)} \rVert_1}{|n|^p}
    \end{equation}
\end{lemma}


\begin{corollary}
    Let $f$ be a function that has $p$ continuous derivatives. Suppose that we wish to bound the truncation error $\lVert f - \hat{f}_N \rVert_2^2 \leq \varepsilon$. Then, the following condition suffices:
    
    \begin{equation}
        N \geq \sqrt[2p-1]{\frac{2\lVert f^{(p)} \rVert_1^2}{(2p - 1)\varepsilon}}
    \end{equation}
    
    The following proof was based on the Giardina and Chirlian proof of a similar result for functions of bounded variation.
\end{corollary}
\begin{proof}
    We start by rewriting the error calculation we found previously:
    \begin{align}
        \lVert f - \hat{f}_N \rVert_2^2
        &= \lVert f \rVert_2^2 - \lVert \hat{f}_N \rVert_2^2 \\
        &= L \sum_{n=-\infty}^{\infty} (|c_n|^2 - |\hat{c}_n|^2) \\
        &= L \sum_{n=N+1}^{\infty} (|c_n|^2 + |c_{-n}|^2)
    \end{align}
    Then, we apply the lemma in order to find a bound.
    \begin{equation}
        \lVert f - \hat{f}_N \rVert_2^2 \leq 2L  \sum_{n=N+1}^{\infty} \frac{\lVert f^{(p)} \rVert_1^2}{n^{2p}}
    \end{equation}
    Since $n^{-2p}$ is monotonically decreasing for $p \geq 0$, we can convert this into an integral inequality.
    \begin{equation}
        2L \sum_{n=N+1}^{\infty} \frac{\lVert f^{(p)} \rVert_1^2}{n^{2p}} \leq 2L \int_{N}^{\infty} \frac{\lVert f^{(p)} \rVert_1^2}{x^{2p}}\ dx 
    \end{equation}
    Simplifying, we obtain:
    \begin{equation}
        \lVert f - \hat{f}_N \rVert_2^2 \leq \frac{2L \lVert f^{(p)} \rVert_1^2}{(2p - 1)N^{2p - 1}}
    \end{equation}
    Finally, we solve for $N$ by upper bounding our entire expression by $\varepsilon$.
    \begin{equation}
        \sqrt[2p-1]{\frac{2L \lVert f^{(p)} \rVert_1^2}{(2p - 1)\varepsilon}} \leq N
    \end{equation}
\end{proof}

\begin{corollary}
    Giardina and Chilian proved that:\\
    Let $f$ be a continuous, bounded function such that $\sup_x |f(x)| = B < \infty$. Suppose that we wish to bound the truncation error $\lVert f -\hat{f}_N \rVert_2^2 \leq \varepsilon$.  Then, the following condition suffices:
    \begin{equation}
        N \geq \frac{2L B^2}{\varepsilon}
    \end{equation}
\end{corollary}

\section{References}

\begin{enumerate}
    \item C. Giardina and P. Chirlian, "Bounds on the truncation error of periodic signals," in IEEE Transactions on Circuit Theory, vol. 19, no. 2, pp. 206-207, March 1972, doi: 10.1109/TCT.1972.1083433.
\end{enumerate}

\end{document}
