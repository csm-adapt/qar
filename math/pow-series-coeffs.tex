\documentclass{article}

\usepackage[utf8]{inputenc}
\usepackage[margin=1in]{geometry}
\usepackage{amsfonts}
\usepackage{amsmath}
\usepackage{mathtools}
\usepackage{hyperref}
\hypersetup{
    colorlinks=true,
    urlcolor=blue,
}

\title{Power Expansion of an Analytic Function \\ of a Variable with Linear Transformation}
\author{Matthew Miller}
\date{\today}

\begin{document}

\maketitle

\section{Scalar Transformation}

\subsection{Notation}
\noindent
Suppose that $x \in \mathbb{R}$ is a scalar, $y \in \mathbb{R}$ is a scalar, $\theta \in \mathbb{R}$ is a scalar, and $\sigma: \mathbb{R} \to \mathbb{R}$ is an analytic function. Since $\sigma$ is analytic, it can be represented as

\begin{align*}
    \sigma(x)
    &= s_0 + s_1 x + s_2 x^2 + \cdots \\
    &= \sum_{k=0}^{\infty} s_{k} x^{k}.
\end{align*}

\noindent
Suppose that we can represent $x$ as a power series of another scalar $z$.

\begin{align*}
    x
    &= a_0 + a_1 z + a_2 z^2 + \cdots \\
    &= \sum_{n=0}^{\infty} a_{n} z^{n}.
\end{align*}

\noindent
Finally, suppose that we have the relation

\begin{equation*}
    y = \sigma(\theta x)
\end{equation*}

\subsection{Objective}
\noindent
We want to represent $y$ as a power series of $z$.

\begin{align*}
    y
    &= b_0 + b_1 z + b_2 z^2 + \cdots \\
    &= \sum_{n=0}^{\infty} b_{n} z^{n}
\end{align*}

\noindent
We will rewrite the expression for $y$ to find these coefficients $b_i$ for $i = 0, 1, 2, \ldots$

\begin{align*}
    y
    &= \sigma(\theta x) \\
    &= \sum_{k=0}^{\infty} s_k (\theta x)^k \\
    &= \sum_{k=0}^{\infty} s_k \left(\theta \sum_{n=0}^{\infty} a_{n} z^{n}\right)^{k} \\
    &= \sum_{k=0}^{\infty} s_k \theta^k \left(\sum_{n=0}^{\infty} a_{n} z^{n}\right)^{k} \\
    &= \sum_{k=0}^{\infty} s_k \theta^k \left(\sum_{k_0 + k_1 + k_2 + \cdots = k} \binom{k}{k_0, k_1, k_2, \cdots} \prod_{n=0}^{\infty} (a_n z^{n})^{k_n} \right)
\end{align*}

\subsection{Coefficient Extraction}

\noindent
To extract the coefficient $b_i$ from the previous equation, we must find terms satisfying 

\begin{align*}
    i
    &= 0 k_0 + 1 k_1 + 2 k_2 + \cdots \\
    &= \sum_{n=0}^{\infty} n k_n
\end{align*}

\noindent
and

\begin{align*}
    k
    &= k_0 + k_1 + k_2 + \cdots \\
    &= \sum_{n=0}^{\infty} k_n
\end{align*}

\noindent
If satisfying both of these conditions, then the value

\begin{equation*}
    s_k \theta^k \binom{k}{k_0, k_1, k_2, \cdots} \prod_{n=0}^{\infty} a_n^{k_n}
\end{equation*}

\noindent
is added to the sum of $b_i$. In concise terms we formulate

\begin{align*}
    b_i
    &= \sum_{k=0}^{\infty} s_k \theta^k \sum_{\substack{k_0 + k_1 + k_2 + \cdots = k \\ 0 k_0 + 1 k_1 + 2 k_2 + \cdots = i}} \binom{k}{k_0, k_1, k_2, \cdots} \prod_{n=0}^{\infty} a_n^{k_n} \\
    &= \sum_{k=0}^{\infty} s_k \theta^k \sum_{\substack{k_0 + k_1 + k_2 + \cdots + k_i = k \\ 0 k_0 + 1 k_1 + 2 k_2 + \cdots + i k_i = i}} \binom{k}{k_0, k_1, k_2, \cdots, k_i} \prod_{n=0}^{\infty} a_n^{k_n} 
\end{align*}

\subsection{Worked Coefficients}

\noindent
We can find $b_0$. $i = 0 \implies k_1 = k_2 = \cdots = 0$ so $k_0 = k$. Therefore,

\begin{align*}
    b_0 
    &= \sum_{k=0}^{\infty} s_k \theta^k \sum_{\substack{k_0 + k_1 + k_2 + \cdots = k \\ 0 k_0 + 1 k_1 + 2 k_2 + \cdots = 0}} \binom{k}{k_0, k_1, k_2, \cdots} \prod_{n=0}^{\infty} a_n^{k_n} \\
    &= \sum_{k=0}^{\infty} s_k \theta^k \binom{k}{k} a_0^{k} \\
    &= \sum_{k=0}^{\infty} s_k \theta^k a_0^k
\end{align*}

\noindent
We can also find $b_1$. $i = 1 \implies k_1 = 1, k_2 = k_3 = \cdots = 0$ so $k_0 = k - 1$ and $k_1 = 1$. Therefore,

\begin{align*}
    b_1
    &= \sum_{k=0}^{\infty} s_k \theta^k \sum_{\substack{k_0 + k_1 + k_2 + \cdots = k \\ 0 k_0 + 1 k_1 + 2 k_2 + \cdots = 1}} \binom{k}{k_0, k_1, k_2, \cdots} \prod_{n=0}^{\infty} a_n^{k_n} \\
    &= \sum_{k=0}^{\infty} s_k \theta^k \binom{k}{k - 1, 1} a_0^{k - 1} a_1 \\
    &= \sum_{k=0}^{\infty} k s_k \theta^k a_0^{k-1} a_1
\end{align*}

\noindent
We can also find $b_2$, but the formulation is slightly more complicated. 

\begin{itemize}
    \item $i = 2 \implies k_3 = k_4 = k_5 = \cdots = 0$ and
    \begin{itemize}
        \item $k_2 = 0, k_1 = 2, k_0 = k - 2$ or
        \item $k_2 = 1, k_1 = 0, k_0 = k - 1$
    \end{itemize}
\end{itemize}

\noindent
Therefore,

\begin{align*}
    b_2
    &= \sum_{k=0}^{\infty} s_k \theta^k \sum_{\substack{k_0 + k_1 + k_2 + \cdots = k \\ 0 k_0 + 1 k_1 + 2 k_2 + \cdots = 2}} \binom{k}{k_0, k_1, k_2, \cdots} \prod_{n=0}^{\infty} a_n^{k_n} \\
    &= \sum_{k=0}^{\infty} s_k \theta^k \left(\binom{k}{k - 2, 2, 0}a_0^{k-2} a_1^{2} + \binom{k}{k - 1, 0, 1}a_0^{k-1} a_2\right) \\
    &= \sum_{k=0}^{\infty} s_k \theta^k \left(\frac{k(k-1)}{2} a_0^{k-2}a_1^{2} + k a_0^{k-1}a_2\right)
\end{align*}

\noindent
Note that when calculating $b_i$, the value of $i$ constrains $k_{i+1} = k_{i+2} = \cdots = 0$. Also, $k_1, k_2 \cdots, k_i$ are finite and $k_0 = k - k_1 - k_2 - \cdots - k_i$.

\section{Vector Transformation}
\subsection{Notation}

\noindent
Suppose that $\mathbf{x} \in \mathbb{R}^d, (d \in \mathbb{Z})$ is a vector, $\mathbf{y} \in \mathbb{R}^c, (c \in \mathbb{Z})$ is a vector, $\mathbf{\Theta} \in \mathbb{R}^{c \times d}$ is a matrix, and $\sigma: \mathbb{R} \to \mathbb{R}$ is an analytic function. Since $\sigma$ is analytic, it can be represented as

\begin{align*}
    \sigma(x)
    &= s_0 + s_1 x + s_2 x^2 + \cdots \\
    &= \sum_{k=0}^{\infty} s_{k} x^{k}.
\end{align*}

\noindent
Suppose that we can represent each entry of $\mathbf{x}$ as a power series of the entries of a vector $\mathbf{z} \in \mathbb{R}^{w}, (w \in \mathbb{Z})$.

\begin{align*}
    \forall i &= 1, \cdots, d, \\ x_i
    &= a^{(i)}_{0,0,\cdots,0} + a^{(i)}_{1,0,\cdots,0} z_1 + a^{(i)}_{2,0,\cdots,0} z_1^2 + \cdots \\
    &+ a^{(i)}_{0,1,\cdots,0} z_2 + a^{(i)}_{1,1,\cdots,0} z_1 z_2 + a^{(i)}_{2,1,\cdots,0} z_1^2 z_2 + \cdots \\
    &+ a^{(i)}_{0,0,\cdots,1} z_w + a^{(i)}_{1,0,\cdots,1} z_1 z_w + a^{(i)}_{2,0,\cdots,1} z_1^2 z_w + \cdots \\
    &= \sum_{n_1,n_2,\cdots,n_w}^{\infty,\infty,\cdots,\infty} a^{(i)}_{n_1,n_2,\cdots,n_w} z_1^{n_1} z_2^{n_2} \cdots z_w^{n_w}.
\end{align*}

\noindent
Finally, suppose that we have the relation

\begin{equation*}
    \mathbf{y} = \circ\sigma(\mathbf{\Theta} \mathbf{x})
\end{equation*}

\noindent
where $\circ\sigma: \mathbb{R}^{c} \to \mathbb{R}^{c}$ is the $\sigma$ function applied element-wise to a vector.

\subsection{Objective}
\noindent
We want to represent each entry of $\mathbf{y}$ as a power series of the entries of $\mathbf{z}$.

\begin{align*}
    \forall i &= 1, \cdots, c, \\ y_i
    &= b^{(i)}_{0,0,\cdots,0} + b^{(i)}_{1,0,\cdots,0} z_1 + b^{(i)}_{2,0,\cdots,0} z_1^2 + \cdots \\
    &+ b^{(i)}_{0,1,\cdots,0} z_2 + b^{(i)}_{1,1,\cdots,0} z_1 z_2 + b^{(i)}_{2,1,\cdots,0} z_1^2 z_2 + \cdots \\
    &+ b^{(i)}_{0,0,\cdots,1} z_w + b^{(i)}_{1,0,\cdots,1} z_1 z_w + b^{(i)}_{2,0,\cdots,1} z_1^2 z_w + \cdots \\
    &= \sum_{n_1,n_2,\cdots,n_w}^{\infty,\infty,\cdots,\infty} b^{(i)}_{n_1,n_2,\cdots,n_w} z_1^{n_1} z_2^{n_2} \cdots z_w^{n_w}.
\end{align*}

\noindent
We will rewrite the expression for $y_i$ to find these coefficients $b^{(i)}_{n_1, n_2, \cdots, n_w}$ for $(n_1, n_2, \cdots, n_w)$ from $(0, 0, \cdots, 0)$ to $(\infty, \infty, \cdots, \infty)$.

\begin{align*}
    \forall i &= 1, \cdots, c, \\ y_i
    &= \left[\circ\sigma(\mathbf{\Theta} \mathbf{x})\right]_i \\
    &= \sigma(\left[\mathbf{\Theta} \mathbf{x}\right]_i) \\
    &= \sigma(\mathbf{\theta}_i \mathbf{x}) \\
    &= \sum_{k=0}^{\infty} s_k (\mathbf{\theta}_i \mathbf{x})^k \\
    &= \sum_{k=0}^{\infty} s_k \left(\sum_{j=1}^{d} \theta_{ij} x_{j}\right)^k \\
    &= \sum_{k=0}^{\infty} s_k \left(\sum_{k_1 + \cdots + k_d = k} \binom{k}{k_1, \cdots, k_d} \prod_{j=1}^{d} (\theta_{ij} x_j)^{k_j} \right) \\
    &= \sum_{k=0}^{\infty} s_k \left(\sum_{k_1 + \cdots + k_d = k} \binom{k}{k_1, \cdots, k_d} \prod_{j=1}^{d}\theta_{ij}^{k_j} \left(\sum_{n_1, \cdots, n_w}^{\infty,
    \cdots, \infty} a^{(j)}_{n_1,\cdots,n_w} z_1^{n_1}  \cdots z_w^{n_w} \right)^{k_j}\right) \\
    &= \sum_{k=0}^{\infty} s_k \left(\sum_{k_1 + \cdots + k_d = k} \binom{k}{k_1, \cdots, k_d} \prod_{j=1}^{d} \theta_{ij}^{k_j} \left(\sum_{\sum [l] = k_j} \binom{k_j}{[l]} \prod_{n_1, \cdots, n_w}^{\infty, \cdots, \infty} (a^{(j)}_{n_1, \cdots, n_w} z_1^{n_1} \cdots z_w^{n_w})^{l_{n_1, \cdots, n_w}} \right)\right)
\end{align*}

\noindent
where

\begin{align*}
    [l] = l_{\underbrace{0, \cdots, 0}_{\times w}}, \ldots, l_{\underbrace{\infty, \cdots, \infty}_{\times w}}
\end{align*}

\subsection{Coefficient Extraction}

\noindent
To extract the coefficient $b^{(i)}_{m_1, \cdots, m_w}$ from the previous equation, we must find terms satisfying index constraints

\begin{align*}
    k_1 + k_2 + \cdots + k_d &= k \\
    l_{0, \cdots, 0} + \cdots + l_{\infty, \cdots, \infty} &= k_j
\end{align*}

\noindent
and power constraints

\begin{align*}
    \sum_{n_1, \cdots, n_w}^{\infty, \cdots, \infty} n_1 l_{n_1, \cdots, n_w} &= m_1 \\
    \sum_{n_1, \cdots, n_w}^{\infty, \cdots, \infty} n_2 l_{n_1, \cdots, n_w} &= m_2 \\
    \vdots \qquad &= \;\; \vdots \\
    \sum_{n_1, \cdots, n_w}^{\infty, \cdots, \infty} n_w l_{n_1, \cdots, n_w} &= m_w.
\end{align*}

\noindent
Notice that the power constraints can be simplified slightly since any solution must also satisfy

\begin{equation*}
    l_{n_1, \cdots, n_{p-1}, m_p + 1, n_{p+1}, \cdots, n_w} = l_{n_1, \cdots, n_{p-1}, m_p + 2, n_{p+1}, \cdots, n_w} = \cdots = 0
\end{equation*}

\noindent
resulting in

\begin{align*}
    \sum_{n_1, \cdots, n_w}^{m_1, \cdots, m_w} n_1 l_{n_1, \cdots, n_w} &= m_1 \\
    \sum_{n_1, \cdots, n_w}^{m_1, \cdots, m_w} n_2 l_{n_1, \cdots, n_w} &= m_2 \\
    \vdots \qquad &= \;\; \vdots \\
    \sum_{n_1, \cdots, n_w}^{m_1, \cdots, m_w} n_w l_{n_1, \cdots, n_w} &= m_w.
\end{align*}

\noindent
If satisfying all of these constrains, then the value

\begin{equation*}
    s_k \binom{k}{k_1, \cdots, k_d} \binom{k_j}{[l]} \theta_{ij}^{k_j} (a^{(j)}_{n_1, \cdots, n_w})^{l_{n_1, \cdots, n_w}}
\end{equation*}

\noindent
is added to the sum of $b^{(i)}_{m_1, \cdots, m_w}$. In concise terms we formulate

\begin{align*}
    b^{(i)}_{m_1, \cdots, m_w} = \sum_{k=0}^{\infty} s_k \sum_{k_1 + \cdots + k_d = k} \binom{k}{k_1, \cdots, k_d} \prod_{j=1}^{d} \theta_{ij}^{k_j} \sum_{\substack{l_{0, \cdots, 0} + \cdots + l_{\infty, \cdots, \infty} = k_j \\ \sum_{n_1, \cdots, n_w}^{m_1, \cdots, m_w} n_1 l_{n_1, \cdots, n_w} = m_1 \\ \vdots \\ \sum_{n_1, \cdots, n_w}^{m_1, \cdots, m_w} n_w l_{n_1, \cdots, n_w} = m_w}} \binom{k_j}{[l]} (a^{(j)}_{n_1, \cdots, n_w})^{l_{n_1, \cdots, n_w}}
\end{align*}

\noindent
The only issue is that of finding solutions to the constraints.

\subsection{Interpretation}

We can use this method for deriving power series for repeatedly composed functions with linear transformations (i.e. $\mathbf{y} = f(\mathbf{\Theta}_3 f(\mathbf{
\Theta}_1 f(\mathbf{\Theta}_1 \mathbf{x})))$). We represent our original input variable as $\mathbf{x} = \mathbf{z}$ so that we have coefficients $a_{1, 0, \cdots, 0} = a_{0, 1, \cdots, 0} = \cdots = a_{0, 0, \cdots, 1} = 1$ with all other coefficients zero. Then, we update $\mathbf{a}$ each composition of the function to obtain $\mathbf{b}$ for $\mathbf{y}$. This can be easily applied to neural networks.

\vspace{0.10in}
\noindent
In order to make this method computationally feasible, we must truncate the power series at some precision. Say the maximum power we wish on any variable is $K$, then, we simply replace instances of $\sum_{k=0}^{\infty} s_k x^k$ with $\sum_{k=0}^{K} s_k x^k$. This works because our power constraints limit the maximum powers of our variable. Thus, coefficients of order $0, 1, \cdots, K$ will have no error and our approximation will only have truncation error from dropping terms with order $> K$.

\subsection{Constraint Analysis}

The constraints are of a form of a restricted partition within number theory. There is literature concerning these exact constraints (see \href{https://arxiv.org/pdf/1611.09931.pdf}{Gaussian Polynomials and
Restricted Partition Functions with Constraints}). We can construct an algorithm that operates in $\mathcal{O}(K^w)$ time by looping each indices $l_{0, \cdots, 0}, \cdots, l_{m_1, \cdots, m_w}$ from $0$ to $K$ and checking if they satisfy the constraints. This warrants some future study to see if there are ways to reduce the operation time and find solutions to the system of constraints more quickly.

\end{document}