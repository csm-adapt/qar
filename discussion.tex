\section{Discussion}\label{discussion}

Many non-trivial problems in materials science, and in science more broadly, are explained not through a single constitutive relationship, but through a superposition of contributing physics. 

Figure~\ref{fig:stress} shows an artificial dataset constructed to replicate the impact of yield stress in a two-phase, solid-solution strengthened alloy system. Using a combination of composite theory for the contribution of flow stress, {\color{red} NAME} solid solution \cite{solid solution}, and Hall-Petch \cite{Hall-Petch} strengthening, the expected yield stress is

\begin{equation}
	\sigma_y = F_v^A \sigma_f^A + F_v^B \sigma_f^B + \sum_i C_i [x_i]^{2/3} + \sum_j k_j d_j^{-1/2} + \ldots
\end{equation}

\noindent with free parameters \\[2ex]
\begin{tabular}{l l}
	$F_v^i$	& Volume fraction of phase $i$ \\
	$[x_i]$	& Concentration of solute $i$ \\
	$d_j$	& Average grain diameter of phase $j$
\end{tabular}
\\[2ex]
\noindent and fixed parameters \\[2ex]
\begin{tabular}{l l}
	$\sigma_f^i$	& Flow stress of phase $i$ \\
	$C_i$		& Solid solution strengthening coefficient for solute species $i$ \\
	$k_j$		& Hall-Petch strengthening coefficient for phase $j$
\end{tabular}
\\[2ex]

The goal is to iteratively improve on this constitutive model one term at a time, and monitor the effect on the residuals between the predicted yield, $\hat{\sigma_y}$ and the actual yield $\sigma_y$.

The analytical form, combining Equations~(\ref{eqn:nn analytical form}) and (\ref{eqn:sigmoid zeta expansion}), the estimated output of a two-layer NN can be written as
%\begin{eqnarray}
%	\widehat{\bold y} & = & \frac{1}{2} - 2 \sum_{m=1}^\infty b_m \left( 2\boldsymbol\theta_2 \left( \frac{1}{4} - \sum_{n=1}^\infty a_n (\boldsymbol\theta_1 {\bold x})^{2n-1} \right) \right)^{2m - 1} \\
%		& = & \frac{1}{2} - 2 \sum_{m=1}^\infty 2^{2m-1} b_m \boldsymbol\theta_2^{2m - 1} \left( \frac{1}{4} - a_1 (\boldsymbol\theta_1 {\bold x}) - a_2 (\boldsymbol\theta_1 {\bold x})^3 - \ldots \right)^{2m - 1} \nonumber \\
%		& = & \frac{1}{2} - \sum_{m=1}^\infty 2^{2m} b_m \boldsymbol\theta_2^{2m - 1} \left(v_0 + v_1 + \ldots \right)^{2m - 1},\ v_0 = \frac{1}{4}, v_1 = -a_1 \boldsymbol\theta_1 {\bold x}, v_2 = \ldots \nonumber \\
%		& = & \frac{1}{2} - \sum_{m=1}^\infty 2^{2m} b_m \boldsymbol\theta_2^{2m - 1} \sum_{k_1 + k_2 + \ldots + k_n = 2m - 1} \binom{2m-1}{k_0, k_1, k_2, \ldots, k_n} v_0^{k_0} v_1^{k_1} \ldots v_n^{k_n} \nonumber \\
%		& = & \frac{1}{2} - \sum_{m=1}^\infty \sum_{k_1 + k_2 + \ldots + k_n = 2m - 1} 2^{2m} b_m \boldsymbol\theta_2^{2m - 1} \binom{2m-1}{k_0, k_1, k_2, \ldots, k_n} v_0^{k_0} v_1^{k_1} \ldots v_n^{k_n}
%\end{eqnarray}
%where
%\[
%	\binom{2m-1}{k_0, k_1, \ldots, k_n} = \frac{(2m-1)!}{k_0! k_1! \ldots k_n!}
%\]
\begin{eqnarray}
	{\bold y}_1 & = & \sum_{k=0}^\infty a_k (\boldsymbol\theta_1^T {\bold x})^k \nonumber \\
	{\bold y}_2 & = & \sum_{k=0}^\infty b_k (\boldsymbol\theta_2^T {\bold y}_1)^k \nonumber \\
		& = & b_0 {\bold 1} + \nonumber\\
		&   & + b_1 (\tilde a_0 +%
					(\tilde a_1 + %
						(\tilde a_2 + %
							(\tilde a_3 + %
								(\ldots) \tilde {\bold x} )%
							\tilde {\bold x} )%
						\tilde {\bold x} )%
					\tilde {\bold x}) \nonumber \\
		&   & + b_2 (\tilde a_0 +%
					(\tilde a_1 + %
						(\tilde a_2 + %
							(\tilde a_3 + %
								(\ldots) \tilde {\bold x} )%
							\tilde {\bold x} )%
						\tilde {\bold x} )%
					\tilde {\bold x})^2 \nonumber \\
		&   & \vdots \nonumber \\
		&   & + b_k (\tilde a_0 +%
					(\tilde a_1 + %
						(\tilde a_2 + %
							(\tilde a_3 + %
								(\ldots) \tilde {\bold x} )%
							\tilde {\bold x} )%
						\tilde {\bold x} )%
					\tilde {\bold x})^k \nonumber \\
		&   & \vdots \nonumber \\
		&   & \text{where}\ \tilde a_i = \boldsymbol\theta_2^T a_i, %
					\ \tilde{\bold x} = \boldsymbol\theta_1^T {\bold x} \nonumber \\
		& = & \sum_{N=0}^\infty %
				b_N %
				\sum_{j=0}^N \binom{N}{j} \tilde a_0^{N-j} %
				\sum_{k=0}^j \binom{j}{k} \tilde a_1^{j-k} %
				\cdots %
				\tilde{\bold x}^{j+k+\cdots} \\
		& = & \sum_{N=0}^\infty %
				\sum_{j=0}^N %
				\sum_{k=0}^j %
				\cdots %
					b_N %
					\binom{N}{j} \left( \boldsymbol\theta_2^T a_0 \right)^{N-j} %
					\binom{j}{k} \left( \boldsymbol\theta_2^T a_1 \right)^{j-k} %
					\cdots %
					\left(\boldsymbol\theta_1^T {\bold x}\right)^{j+k+\cdots} \nonumber \\
		& = & \sum_{N=0}^\infty %
				\sum_{j=0}^N %
				\sum_{k=0}^j %
				\cdots %
					b_N %
					\binom{N}{j} %
					\binom{j}{k} %
					\cdots %
					\left( \boldsymbol\theta_2^T a_0 \right)^{N-j} %
					\left( \boldsymbol\theta_2^T a_1 \right)^{j-k} %
					\cdots %
					\left(\boldsymbol\theta_1^T {\bold x}\right)^{j+k+\cdots} \nonumber \\
		& = & \sum_{k=0}^\infty c_k \left( \boldsymbol\theta_1 x \right)^k
\end{eqnarray}
{\color{red} If $a_i = 0\ \forall\ i\ \text{even}$, then are Equations~(13-14) everywhere 0?}

\noindent {\color{red} How does dropout affect this?}
