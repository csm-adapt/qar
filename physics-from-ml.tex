\documentclass[11pt]{amsart}
\usepackage{geometry}                % See geometry.pdf to learn the layout options. There are lots.
\geometry{letterpaper}                   % ... or a4paper or a5paper or ...
%\geometry{landscape}                % Activate for for rotated page geometry
%\usepackage[parfill]{parskip}    % Activate to begin paragraphs with an empty line rather than an indent
\usepackage{graphicx}
\usepackage{amssymb}
\usepackage{amsthm}
\usepackage{amsmath}
\usepackage{rotating}
\usepackage{epstopdf}
% \usepackage{multirow}
\usepackage{xcolor}
\usepackage{array}
\DeclareGraphicsRule{.tif}{png}{.png}{`convert #1 `dirname #1`/`basename #1 .tif`.png}

% %%%%%%%%%%%%%%%%%%%%%%%%%%%%%%%%%%%%%%%%%%%%%%%%%%%%%%%%%%%%%%%%%%%%%%%%%%%%%
% user-defined environment, operators, commands, etc.
\newenvironment{conditions}
  {\par\vspace{\abovedisplayskip}\renewcommand{\arraystretch}{1.5} \renewcommand{\tabcolsep}{0.2cm}\begin{tabular}{>{$}l<{$} @{${}={}$} l}}
  {\end{tabular}\par\vspace{\belowdisplayskip}}

\DeclareMathOperator*{\argmin}{argmin}
\DeclareMathOperator*{\argmax}{argmax}

\newtheorem{condition}{Condition}

\newcommand{\he}[2][{}]{\ensuremath{{#1}^{\circ #2}}}
\newcommand{\densematrix}[3]{%
    \ensuremath{\begin{pmatrix}
        #1_{11}  & #1_{12}    & \cdots    & #1_{1#3}    \\
        #1_{21}  & #1_{22}    &           & \vdots      \\
        \vdots   &            & \ddots    &             \\
        #1_{#21} & \cdots     &           & #1_{#2#3}
    \end{pmatrix}}}
% %%%%%%%%%%%%%%%%%%%%%%%%%%%%%%%%%%%%%%%%%%%%%%%%%%%%%%%%%%%%%%%%%%%%%%%%%%%%%

\title{A General Approach for Learning Constitutive Relationships (Physical Laws) from Neural Networks}
\author{Andrew Temple}
\author{Aaron Stebner}
\author{Peter Collins}
\author{Branden Kappes}
%\date{}                                           % Activate to display a given date or no date

\begin{document}
\maketitle
%\section{}
%\subsection{}

\section{Introduction}\label{introduction}

With apologies to the Bard of Avon, \\

\vspace{2mm}
\noindent
\textit{Two methods, both alike in dignity, \\
In fair sciences, where we lay our scene. \\
From ancient elegance break to methods mutinous, \\
Where machine learning makes physical laws unclean, \\
From forth the noble loins of these two foes\\
A pair of cross-term'd numbers seek their place;\\
Whose serial expansions, equally covariant\\
Do with their projections, bury their parents strife.\\
Our fearful assessment of their dual-space truth,\\
And the continuance of their parents, sage,\\
Which, for their children's union, sought to achieve,\\
Is now the two hours' traffic of our stage;\\
The which, if you with patent eyes attend,\\
What here shall miss, our toil shall strive to mend.\\}

\vspace{2mm}

With the proliferation of increasingly powerful machine learning strategies, and their promise of the automatic discovery of new physics, phenomena, properties, and correlations, it is seductive to trust machine learning to solve our most complex problems. The unintended consequence of relegating one's data to machine learning is that our native physical understanding of currently-understood phenomenon may atrophy, and our development of descriptions of new physics may stagnate.  These risks could have profound impacts on numerous scientific fields. Consider the instruction of machine learning, where students, and hence future scientists, engineers and educators, may, unintentionally, understand machine learning to be a cudgel against which every problem which has a solution falls. When one tool is so powerful as to render other methods seemingly obsolete, it is natural to ``disarm'' oneself of the tools that seem obsolete.  Yet, such tools are not obsolete.  Various physical phenomenon are exceptionally well described by laws (the laws of Thermodynamics, kinetics, motion, ...) or theories (e.g., dislocations which underpin our understanding of strength in metals and alloys).  Those laws and theories permit us to creatively explore spaces where data does not exist, hypothesize, theorize, and test our hypotheses.  Should mastery of our understanding of physical processes suffer, the scientific method in its currently and widely accepted form, becomes harder to apply.  Thus, we must: retain mastery of current theory; postulate logically new theories; and leverage mathematical powers to understand complex hyperspaces describing n-variables. \\

The problem of whether to apply our most powerful tools in our analytical arsenal to a problem, or rigorously develop theories or calculate previously unknown physical constants seems to become an either-or problem.  Much like the houses of the Montagues and Capulets in Shakespeare, the tools of Machine Learning and Phenomenological Relationships seem irrenconcilable. Yet, the problem lies with neither approach (both alike in dignity), rather it stems from two very human limitations. Firstly, we have great difficulty interpreting problems in higher-order space.  It is easy to understand problems which have a single independent variable against which a phenomena depends, resulting in y-x plots of data.  It is not much harder to understand problems which have two independent variables against which a phenomena depends, as these can be represented in Euclidean space, and with imagination we can understand problems with three independent variables (say x,y and time) against which a phenomena (say, z=f(xy,t)) depends.  Beyond this, we generally lack tools to understand or visualize an n-dimensional hyperspace.  Secondly, we can develop an implicit bias when wrestling with mathematical expressions, and may begin to believe that one (and hence, only one) functional expression can be used to fit a relationship.  We develop mathematical ``favorites'', tools that, quite rationally, we turn to when solving certain problems.  We turn to these expressions for two reasons: they work, and they are reduced to the simplest form possible.  In materials science, against which the framework presented in this paper has been conceived it is common to turn to the Arrhenius relationship for phenomena which have temperature dependencies.  This simple form, $A = A_O\exp{-\frac{Q}{kT}}$, is taught extensively in a wide variety of classes, and yet, there is nothing in nature that dictates the exponential form be used.  It is used because it is the simplest known mathematical expression to fit the observed data.  And yet, it could be expanded in, e.g., polynomial form\ldots
\begin{align}
    A &= \sum_{m=0}^\infty a_m x^m \nonumber\\
        &= \sum_{m=0}^\infty \frac{(-1)^m}{m!} \left(\frac{Q}{kT}\right)^m
\label{eqn:coefficient generating function example}
\end{align}
\noindent where $a_m = (-1)^m/m!$ is the coefficient generating function and $x = Q/kT$ is the non-dimensionalized temperature.

If this recognition that our most trusted expressions can be re-written in another form, then so too must the reduced functions describing the most flexible and powerful neural networks.  The next logical step is to postulate that these expanded forms of different representations of the same physical phenomena can be compared, and assessed for their self-similarity, providing a \textit{translation} between one form and another.  A Rosetta Stone, if you will, where the mathematical functions (i.e., the language) in one reference frame can be translated and assessed for equivalency to the mathematical functions of the second reference frame.

The precursors to this thought process began when H.L. Fraser et al. began conducting so-called virtual experiments, which permitted a well-trained neural network to be probed to determine the influence of one variable on a physical property while all of the other potential independent variables were artificially held at their average values [refs]. These virtual experiments were slices through an n-dimensional hyperspace, and it became obvious that the lower dimension slices could be described using simpler functions than the full neural network.  This early work was built upon the initial efforts by H.K.D.H. Bhadeshia to solve complex materials science problems using artificial neural networks [refs] and the work of D.J.C. MacKay [refs] to incorporate Bayesian statistics and feedback loops.  Following this prior work, with the assumption that a mapping was possible between the basis vectors of the n-variable hyperspace equally represented by the expansion of terms of an artificial neural network and the basis vectors representing the physical processes, Ghamarian and Collins, seeking to establish a constitutive equation for the room temperature yield strength of a titanium alloy given variations in its compositional and microstructural states, applied a hybrid artificial neural network-genetic algorithm method that optimized the unknowns in a postulated physically-based equation, testing the optimized physically-based model against slices through hyperspace function representing the neural network model and the data [refs]. This latter effort was quite inefficient, but resulted in a physically-based equation that was demonstrated, in subsequent work, to be generalized for multiple different variations of processing and with compositional variations [refs].

This previous work strongly suggests that the hypothesis we propose below is valid, and that a more fulsome mathematical treatment is merited and which is the subject of this paper.  We recognize that this approach lies neither fully in a mathematics space nor in an engineering/science space. Consequently, we aim to provide both sufficiency of the derivations and sufficiency of motivation and impact, so that the paper is accessible to readers of various communities.  While we develop the approach based upon a materials science problem, we hope that the general applicability will be apparent, as it is easy to imagine this approach impacting a diverse range of disciplines, including: genetics, public health, biological sciences, earth sciences, information sciences (including signals analysis), physical sciences, applied variants thereof (medicine, environmental activities, engineering), space sciences, and economics.  We finally include an appendix that contains the expansions of the activation functions and the generating functions of common functions that form the basis of some relevant physics.  We recognize that this appendix is far from complete, but hope that the logic presented will permit those interested in identifying and developing additional functions as the specific applications demand.

\begin{hypothesis}
We hypothesize that the physics of an arbitrarily complex process can be extracted by fitting the series expansion of known, postulated, or potential physical relationship to the coefficients of a series expansion of a trained neural network when those series are supported on the same basis.
\end{hypothesis}

Coefficient generating functions, such as that shown in Equation~\ref{eqn:coefficient generating function example}, have been tabulated for a number of common mathematical expressions, c.f. Table~\ref{tab:generating functions of common functions}. For constitutive relationships composed of a superposition of these terms, the series expansion coefficients become an analytical expression in terms of the physical constants of the constitutive relationships: the Hall-Petch coefficient, $k$; the Young's modulus, $E$; etc. In general, we represent these coefficients as $C_k$ throughtout this work. Any physical constants known a priori reduce the number of $C_k$ in the final coefficient generating function expression.


\section{Methods}\label{methods}

Fully dense neural network (NN) architectures, such as the one shown in Figure~\ref{fig:nn-1}, perform a sequence of affine transformations, ${\bold z}_i \leftarrow \boldsymbol\theta_i {\bold x}^{(i)}$, followed by element-wise functional operations, $\sigma({\bold z}_i)$ to introduce non-linearity at each layer; that is, each layer stretches and distorts the underlying space.
\begin{figure}[htbp]
\begin{center}
\includegraphics[width=0.4\textwidth]{fig/neural-network-01}
\caption{Schematic view of a fully dense neural network. Each sequence of affine and non-linear transformations are captured in the function, $f_i({\bold x}): {\bold x}^{(i+1)} \leftarrow \sigma(\boldsymbol\theta_i {\bold x}^{(i)})$}
\label{fig:nn-1}
\end{center}
\end{figure}

A typical neural network, as given in (insert neural network reference), is nothing more than an arbitrary function generator, but at present, the network weights can not map back to analytic forms that capture and describe the underlying physics. There are, however, many such mappings through polynomial series expansions (possible reference to other polynomial series expansion applications?). We hypothesize that the physics of a process can be extracted by fitting the polynomial expansions of known physical relationships to the polynomial coefficients of a polynomial series expansion of Equation~(\ref{eqn:nn analytical form}). \\

The goal is to obtain an activation generating function for the chosen activation function in the neural network. There exists many potential options for choosing an activation function such as softplus, softmax, ReLU (rectified linear units), and logistic sigmoid. The details of the derivations for the generating functions of the common activation functions can be found in the appendix. The generating function for the ReLU is dependent on both the input variables and the network weights. Although ReLU (rectified linear units) have become a more common activation function, its discontinuity at $x = 0$ requires an infinite series to fully capture the behavior at this transition. {\color{red}Note: add more pros and cons about each of the common activation functions if necessary}

%\begin{equation}
%	f(x) = log(1+e^x)
%\end{equation}
%
%The series expansion for the exponential
%\begin{equation}
%	e^x = \sum_{k=0}^\infty \frac{x^k}{k!}
%\end{equation}
%where, $a_k = \frac{1}{k!}$, can be expressed as
%\begin{equation}
%	e^x = \sum_{k=0}^\infty x^k a_k
%\end{equation}
%
%A similar series expansion for the logarithm
%\begin{equation}
%	log(1+x) = \sum_{n=1}^\infty (-1)^{n+1} \frac{x^n}{n}
%\end{equation}
%where, $x = e^x$,
%allows for the softplus function to be represented as
%\begin{equation}
%	f(x) = \sum_{n=1}^\infty \frac{(-1)^{n+1}}{n} (\sum_{k=0}^\infty x^k a_k)^n
%\end{equation}
%
%If the expansion of $e^x$ is performed, then it can be seen that
%\begin{equation}
%	(\sum_{k=0}^\infty x^k a_k)^n = (a_0+x(a_1+x(a_2+xa_3...)))^n
%\end{equation}

%Beginning of old softplus derivation and could probably be deleted
%However, the softplus function,
%
%\begin{equation}
%	f(x) = log(1+e^x)
%\end{equation}
%
%The series expansion for the exponential
%\begin{equation}
%	e^x = \sum_{k=0}^\infty \frac{x^k}{k!}
%\end{equation}
%where, $a_k = \frac{1}{k!}$, can be expressed as
%\begin{equation}
%	e^x = \sum_{k=0}^\infty x^k a_k
%\end{equation}
%
%A similar series expansion for the logarithm
%\begin{equation}
%	log(1+x) = \sum_{n=1}^\infty (-1)^{n+1} \frac{x^n}{n}
%\end{equation}
%where, $x = e^x$,
%allows for the softplus function to be represented as
%\begin{equation}
%	f(x) = \sum_{n=1}^\infty \frac{(-1)^{n+1}}{n} 				(\sum_{k=0}^\infty x^k a_k)^n
%\end{equation}
%
%If the expansion of $e^x$ is performed, then it can be seen that
%\begin{equation}
%	(\sum_{k=0}^\infty x^k a_k)^n = 								(a_0+x(a_1+x(a_2+xa_3...)))^n
%\end{equation}
%
%Using the binomial theorem
%\begin{equation}
%	(a+b)^n = \sum_{m=0}^{n} \binom{n}{m} a^{n-m} b^m
%\end{equation}
%where, $a=a_0$ and $b=(x(a_1+x(a_2+xa_3...)))^n$, the series expansion of the exponential can be represented as
%\begin{equation}
%	(\sum_{k=0}^\infty x^k a_k)^n = \sum_{m=0}^{n} 				\binom{n}{m} a_0^{n-m} (a_1+x(a_2+xa_3...))^n x^n
%\end{equation}
%
%Continuing with this approach, it can be seen that ...
%End of old softplus derivation

\subsection{Iterative Determination of ANN Series Expansion Coefficients}
{\color{red}
Any series expansion description of a deep neural network, which is necessarily multilayered, requires a explicit generating function for the coefficients of each layer; a generating function that is a function only of the coefficients of the previous layer and the coefficient generating function of the series expansion of the current layer's activation function.

A derivation for the polynomial expansion coefficient generating function for vector-valued function (layer 1 to layer 2) is presented below, which is an extension of a scalar expansion, provided in \ref{appendix}.}

% {\color{red}
% The element-wise exponent is used repeatedly in both the constitutive relation and neural network expansions. For the product of two matrices, $A: A \in \mathbb{R}^{m \times n}$ and $B: B \in \mathbb{R}^{n \times q}$, the element-wise exponent results in an expanded basis that explicitly includes the cross terms from ${\bf A}$ and ${\bf B}$.
%
% \begin{subequations}
% \begin{align}
%         \left( AB \right)^{\circ p} &= \he[\left( \densematrix{a}{m}{n} \densematrix{b}{n}{q} \right)]{p} \nonumber \\
%             &= \begin{pmatrix}
%                 {\bf a}_{1k}{\bf b}_{k1}  & {\bf a}_{1k}{\bf b}_{k2}    & \cdots    & {\bf a}_{1k}{\bf b}_{kq}   \\
%                 {\bf a}_{2k}{\bf b}_{k1}  & {\bf a}_{2k}{\bf b}_{k2}    &           & \vdots \\
%                 \vdots                    &                             & \ddots    &          \\
%                 {\bf a}_{mk}{\bf b}_{k1}  & \cdots                      &           & {\bf a}_{mk}{\bf b}_{kq}
%             \end{pmatrix}^{\circ p} \nonumber \\
%             &= \begin{pmatrix}
%                 ({\bf a}_{1k}{\bf b}_{k1})^{p}  & ({\bf a}_{1k}{\bf b}_{k2})^{p}    & \cdots    & ({\bf a}_{1k}{\bf b}_{kq})^{p}   \\
%                 ({\bf a}_{2k}{\bf b}_{k1})^{p}  & ({\bf a}_{2k}{\bf b}_{k2})^{p}    &           & \vdots \\
%                 \vdots                          &                                   & \ddots           &          \\
%                 ({\bf a}_{mk}{\bf b}_{k1})^{p}  & \cdots                            &           & ({\bf a}_{mk}{\bf b}_{kq})^{p}
%             \end{pmatrix} \label{eqn:hadmard exponent matrix} \\[3ex]
%         ({\bf a}_{ik}{\bf b}_{kj})^{p} &= (a_{i1}b_{1j} + a_{i2}b_{2j} + \cdots + a_{in}b_{nj})^p \nonumber \\
%             &= \sum_{k1 + k2 + \cdots + k_n = p} \binom{p}{k_1, k_2, \ldots, k_n} \prod_{m=1}^n (a_{im}b_{mj})^{k_m} \nonumber \\
%             &= \sum_{\| {\bf k} \|_1 = p} \binom{p}{k_1, k_2, \ldots, k_n} \prod_{m=1}^n (a_{im}b_{mj})^{k_m} \\%
%     \label{eqn:hadamard exponent vector}
% \end{align}
% \label{eqn:nondistributive hadamard}
% \end{subequations}
% where summation over a repeated index is assumed. That is, Equation~\ref{eqn:hadamard exponent vector} is the expansion of the element-wise exponent of a vector product, $({\bf a}{\bf b})\he{p} = ({\bf a}{\bf b})^p \ne \he[{\bf a}]{p}\he[{\bf b}]{p}$.
% }
%
% % %%%%%%%%%%%%%%%%%%%%%%%%%%%%%%%%%%%%%%%%%%%%%%%%%%%%%%%%%%%%%%%%%%%%%%%%%%%%
% The analytical form, combining Equations~(\ref{eqn:nn analytical form}) and (\ref{eqn:sigmoid zeta expansion}), the estimated output of a two-layer NN can be written as an expansion:
%
% \begin{eqnarray}
% 	{\bold y}_1 & = & \sum_{k=0}^\infty a_k (\boldsymbol\theta_1^T {\bold x})\he{k} \nonumber \\
% 	{\bold y}_2 & = & \sum_{k=0}^\infty b_k (\boldsymbol\theta_2^T {\bold y}_1)\he{k} \nonumber \\
% 		& = & b_0 {\bold 1} + \nonumber\\
% 		&   & + b_1 (\tilde a_0 +%
% 					(\tilde a_1 + %
% 						(\tilde a_2 + %
% 							(\tilde a_3 + %
% 								(\ldots) \tilde {\bold x} )%
% 							\tilde {\bold x} )%
% 						\tilde {\bold x} )%
% 					\tilde {\bold x}) \nonumber \\
% 		&   & + b_2 (\tilde a_0 +%
% 					(\tilde a_1 + %
% 						(\tilde a_2 + %
% 							(\tilde a_3 + %
% 								(\ldots) \tilde {\bold x} )%
% 							\tilde {\bold x} )%
% 						\tilde {\bold x} )%
% 					\tilde {\bold x})^2 \nonumber \\
% 		&   & \vdots \nonumber \\
% 		&   & + b_k (\tilde a_0 +%
% 					(\tilde a_1 + %
% 						(\tilde a_2 + %
% 							(\tilde a_3 + %
% 								(\ldots) \tilde {\bold x} )%
% 							\tilde {\bold x} )%
% 						\tilde {\bold x} )%
% 					\tilde {\bold x})^k \nonumber \\
% 		&   & \vdots
% \end{eqnarray}
% \noindent where $\tilde a_i = \boldsymbol\theta_2^T a_i$ and $\tilde{\bold x} = \boldsymbol\theta_1^T {\bold x}$. All $\boldsymbol\theta_i$, ${\bold x}$, and ${\bold y}$ are augmented to include the bias, ${\bold b}_i$, that is,
% \begin{eqnarray}
% 	{\bold x}&:& {\bold x} \leftarrow \begin{pmatrix}
% 								1 \\
% 								{\bold x}
% 							\end{pmatrix} = \begin{pmatrix}
% 											1 \\
% 											x_0 \\
% 											x_1 \\
% 											\vdots \\
% 											x_n
% 										\end{pmatrix}\\
% 	{\bold y}&:& {\bold y} \leftarrow \begin{pmatrix}
% 								1 \\
% 								{\bold y}
% 							\end{pmatrix} \\
% 	\boldsymbol\theta_i&:& \boldsymbol\theta_i \leftarrow \begin{pmatrix} {\bold b}_i & \boldsymbol\theta_i \end{pmatrix}
% \end{eqnarray}
%
% The element-wise exponent operator, ${\bf x}\he{m} = (x_1^m\ x_2^m\ \cdots\ x_n^m)^T$, raises each element of a vector or matrix to the specified power, $m$.
%
% From Equation~(\ref{eqn:sigmoid zeta expansion}), $a_i  = 0\ \text{for}\ i = 2, 4, 6, \ldots$, and therefore,
% \begin{align}
% 	{\bold y}_1 =& \sum_{k=0}^\infty a_k (\boldsymbol\theta_1^T {\bold x})\he{k} \nonumber \\
% 	{\bold y}_2 =& \sum_{k=0}^\infty b_k (\boldsymbol\theta_2^T {\bold y}_1)\he{k} \nonumber \\
% 		=& b_0 {\bold 1} + \nonumber\\
% 		 &+ b_1 (\tilde a_0 +%
% 					(\tilde a_1 + %
% 						(\tilde a_3 + %
% 							(\tilde a_5 + %
% 								(\ldots) \tilde {\bold x}\he{2} )%
% 							\tilde {\bold x}\he{2} )%
% 						\tilde {\bold x}\he{2} )%
% 					\tilde {\bold x}) \nonumber \\
% 		&+ b_2 (\tilde a_0 +%
% 					(\tilde a_1 + %
% 						(\tilde a_3 + %
% 							(\tilde a_5 + %
% 								(\ldots) \tilde {\bold x}\he{2} )%
% 							\tilde {\bold x}\he{2} )%
% 						\tilde {\bold x}\he{2} )%
% 					\tilde {\bold x})\he{2} \nonumber \\
% 		& \vdots \nonumber \\
% 		& + b_k (\tilde a_0 +%
% 					(\tilde a_1 + %
% 						(\tilde a_3 + %
% 							(\tilde a_5 + %
% 								(\ldots) \tilde {\bold x}\he{2} )%
% 							\tilde {\bold x}\he{2} )%
% 						\tilde {\bold x}\he{2} )%
% 					\tilde {\bold x})\he{k} \nonumber \\
% 		& \vdots \nonumber \\
% 	{\bold y}_2 = & \sum_{N=0}^\infty \sum_{k=0}^{N} \sum_{l=0}^{k} \sum_{m=0}^{l} \ldots %
% 		b_N %
% 		\binom{N}{k,l,m,\ldots} %
% 		\tilde a_0^k \tilde a_1^l \tilde a_3^m \ldots %
% 		\tilde {\bold x}\he{(N-k\ldots)} %
% 		({\tilde {\bold x}^2})\he{(N-k-l\ldots)} %
% 		({\tilde {\bold x}^2})\he{(N-k-l-m\ldots)} \nonumber \\
% 	 =& \sum_{N=0}^\infty \sum_{k=0}^{N} \sum_{l=0}^{k} \sum_{m=0}^{l} \ldots %
% 		b_N %
% 		\binom{N}{k,l,m,\ldots} %
% 		\tilde a_0^k \tilde a_1^l \tilde a_3^m \ldots %
% 		\tilde {\bold x}\he{(l+m+n+\ldots)} %
% 		({\tilde {\bold x}\he{2}})\he{(m+n+\ldots)} %
% 		({\tilde {\bold x}\he{2}})\he{(n+\ldots)}
% 	\label{eqn:ANN power series coefficient generating function}
% \end{align}
% \noindent where $k+l+m+n+\ldots = N$. Collecting coefficients and terms of power $k$,
% \begin{equation*}
% 	{\bold y_2} =  \sum_{k=0}^\infty c_k \tilde{\bold x}\he{k}
% \end{equation*}
% \noindent that, having the same form as Equation~(\ref{eqn:sigmoid zeta expansion}) creates a sequential process for determining the coefficients of the power series expansion of each layer in an ANN. Importantly, the output layer in a ANN regression is a single node with a linear activation, so the final layer, $y_f$, working from the last hidden layer, ${\bold y}_n$, is simply,
% \begin{equation}
% 	y_f = \boldsymbol\theta_n^T {\bold y}_n
% \end{equation}

%Beginning of vector coefficients derivation.%
\subsection{Polynomial Expansion of a Vector Layer}
\label{sec:polynomial series vector}

We can represent the recursive structure of an ANN as a series of transformations on a particular power series. Using this method, we are able to write an algorithm that will compute the coefficients of a power series determined by some ANN up to an arbitrary order of approximation.\\

Here, we give a derivation of the generalized method applicable to any feed-forward ANN.
We present a special case of this derivation where each layer of the ANN has a single neuron in Section \ref{sec:polynomial series scalar}.

{\color{red} For the sake of illustration, we present the next derivation in terms of a simplified ANN with one neuron in each layer. For the treatment of a unsimplified ANN, see Appendix.}

\subsubsection{Notation}

Suppose that $\mathbf{x} \in \mathbb{R}^d, (d \in \mathbb{Z})$ is a vector, $\mathbf{y} \in \mathbb{R}^c, (c \in \mathbb{Z})$ is a vector, $\mathbf{\Theta} \in \mathbb{R}^{c \times d}$ is a matrix, and $\sigma: \mathbb{R} \to \mathbb{R}$ is an analytic function. Since $\sigma$ is analytic, it can be represented as

\begin{align}
    \sigma(x)
    &= s_0 + s_1 x + s_2 x^2 + \cdots \nonumber \\
    &= \sum_{k=0}^{\infty} s_{k} x^{k}.
    \label{eqn:vector layer activation}
\end{align}

Suppose that we can represent each entry of $\mathbf{x}$ as a power series of the entries of a vector $\mathbf{z} \in \mathbb{R}^{w}, (w \in \mathbb{Z})$.

\begin{align}
    \forall i &= 1, \cdots, d, \nonumber \\ x_i
    &= a^{(i)}_{0,0,\cdots,0} + a^{(i)}_{1,0,\cdots,0} z_1 + a^{(i)}_{2,0,\cdots,0} z_1^2 + \cdots \nonumber \\
    &+ a^{(i)}_{0,1,\cdots,0} z_2 + a^{(i)}_{1,1,\cdots,0} z_1 z_2 + a^{(i)}_{2,1,\cdots,0} z_1^2 z_2 + \cdots \nonumber \\
    &+ a^{(i)}_{0,0,\cdots,1} z_w + a^{(i)}_{1,0,\cdots,1} z_1 z_w + a^{(i)}_{2,0,\cdots,1} z_1^2 z_w + \cdots \nonumber \\
    &= \sum_{n_1,n_2,\cdots,n_w}^{\infty,\infty,\cdots,\infty} a^{(i)}_{n_1,n_2,\cdots,n_w} z_1^{n_1} z_2^{n_2} \cdots z_w^{n_w}.
    \label{eqn:vector x series}
\end{align}

Finally, suppose that we have the relation

\begin{equation}
    \mathbf{y} = \mathbf{\sigma}(\mathbf{\Theta} \mathbf{x})
    \label{eqn:vector layer relation}
\end{equation}

where $\sigma$ is being applied element-wise to a vector.\\

Similar to the scalar case, this represents a multi-neuron layer of an ANN with input $\mathbf{x}$, weights $\mathbf{\theta}$, output $\mathbf{y}$, and activation $\sigma$. The value of $z$ can be intrepreted as the input to the entire ANN.

\subsubsection{Objective}

We wish to represent each entry of $\mathbf{y}$ as a power series of the entries of $\mathbf{z}$. That is,

\begin{align}
    \forall i &= 1, \cdots, c, \nonumber \\ y_i
    &= b^{(i)}_{0,0,\cdots,0} + b^{(i)}_{1,0,\cdots,0} z_1 + b^{(i)}_{2,0,\cdots,0} z_1^2 + \cdots \nonumber \\
    &+ b^{(i)}_{0,1,\cdots,0} z_2 + b^{(i)}_{1,1,\cdots,0} z_1 z_2 + b^{(i)}_{2,1,\cdots,0} z_1^2 z_2 + \cdots \nonumber \\
    &+ b^{(i)}_{0,0,\cdots,1} z_w + b^{(i)}_{1,0,\cdots,1} z_1 z_w + b^{(i)}_{2,0,\cdots,1} z_1^2 z_w + \cdots \nonumber \\
    &= \sum_{n_1,n_2,\cdots,n_w}^{\infty,\infty,\cdots,\infty} b^{(i)}_{n_1,n_2,\cdots,n_w} z_1^{n_1} z_2^{n_2} \cdots z_w^{n_w}.
    \label{eqn:vector y series}
\end{align}

Rewriting Equation (\ref{eqn:vector layer relation}) in terms of Equations (\ref{eqn:vector layer activation}) and (\ref{eqn:vector x series}), we obtain

{\color{red} Similarly, $y$ can also be represented as a power series with $b_i$ for $i = 0, 1, 2, \ldots$ as the coefficient using the same approach. We will rewrite the relation for $y$ given by Equation (\ref{eqn:scalar layer relation}) to find the coefficients $b_i$ for $i = 0, 1, 2, \ldots$ for Equation (\ref{eqn: eqn:scalar y series}).}

\begin{align}
    \forall i &= 1, \cdots, c, \nonumber \\ y_i
    &= \left[\mathbf{\sigma}(\mathbf{\Theta} \mathbf{x})\right]_i \nonumber \\
    &= \sigma(\mathbf{\theta}_i \mathbf{x}) \nonumber \\
    &= \sum_{k=0}^{\infty} s_k (\mathbf{\theta}_i \mathbf{x})^k \nonumber \\
    &= \sum_{k=0}^{\infty} s_k \left(\sum_{j=1}^{d} \theta_{ij} x_{j}\right)^k \nonumber \\
    &= \sum_{k=0}^{\infty} s_k \left(\sum_{k_1 + \cdots + k_d = k} \binom{k}{k_1, \cdots, k_d} \prod_{j=1}^{d} (\theta_{ij} x_j)^{k_j} \right) \nonumber \\
    &= \sum_{k=0}^{\infty} s_k \left(\sum_{k_1 + \cdots + k_d = k} \binom{k}{k_1, \cdots, k_d} \prod_{j=1}^{d}\theta_{ij}^{k_j} \left(\sum_{n_1, \cdots, n_w}^{\infty,
    \cdots, \infty} a^{(j)}_{n_1,\cdots,n_w} z_1^{n_1}  \cdots z_w^{n_w} \right)^{k_j}\right) \nonumber \\
    &= \sum_{k=0}^{\infty} s_k \left(\sum_{k_1 + \cdots + k_d = k} \binom{k}{k_1, \cdots, k_d} \prod_{j=1}^{d} \theta_{ij}^{k_j} \left(\sum_{\lVert \mathbf{L} \rVert_1 = k_j} \binom{k_j}{\mathbf{L}} \prod_{n_1, \cdots, n_w}^{\infty, \cdots, \infty} (a^{(j)}_{n_1, \cdots, n_w} z_1^{n_1} \cdots z_w^{n_w})^{l_{n_1, \cdots, n_w}} \right)\right). \nonumber \\
    \label{eqn:vector y expansion}
\end{align}

where $\mathbf{L}$ is a collection of non-negative integers that are indices such that

\begin{align*}
    &\mathbf{L} = l_{\underbrace{0, \cdots, 0}_{\times w}}, \ldots, l_{\underbrace{\infty, \cdots, \infty}_{\times w}}, \\
    &\lVert \mathbf{L} \rVert_1 = l_{0, \cdots, 0} + \cdots + l_{\infty, \cdots, \infty}, \\
    &\binom{k_j}{\mathbf{L}} = \binom{k_j}{l_{0, \cdots, 0}, \ldots, l_{\infty, \cdots, \infty}}.
\end{align*}

\subsubsection{Coefficient Extraction}

To find the coefficients $b^{(i)}_{m_1, \cdots, m_w}$ from Equation (\ref{eqn:vector y expansion}), we must find terms satisfying index constraints

\begin{align}
    k &= \sum_{n=1}^{d} k_n \\
    k_j &= \lVert \mathbf{L} \rVert_1
\end{align}

and power constraints

\begin{align}
    m_1 &= \sum_{n_1, \cdots, n_w}^{\infty, \cdots, \infty} n_1 l_{n_1, \cdots, n_w} \nonumber \\
    m_2 &= \sum_{n_1, \cdots, n_w}^{\infty, \cdots, \infty} n_2 l_{n_1, \cdots, n_w} \nonumber \\
    \vdots \;\; &= \qquad \vdots \nonumber \\
    m_3 &= \sum_{n_1, \cdots, n_w}^{\infty, \cdots, \infty} n_w l_{n_1, \cdots, n_w}.
\end{align}

Similar to the scalar case, notice that the power constraints can be simplified since any solution must also satisfy

\begin{equation}
    l_{n_1, \cdots, n_{p-1}, m_p + 1, n_{p+1}, \cdots, n_w} = l_{n_1, \cdots, n_{p-1}, m_p + 2, n_{p+1}, \cdots, n_w} = \cdots = 0.
\end{equation}

resulting in

\begin{align}
    m_1 &= \sum_{n_1, \cdots, n_w}^{m_1, \cdots, m_w} n_1 l_{n_1, \cdots, n_w} \nonumber \\
    m_2 &= \sum_{n_1, \cdots, n_w}^{m_1, \cdots, m_w} n_2 l_{n_1, \cdots, n_w} \nonumber \\
    \vdots \;\; &= \qquad \vdots \nonumber \\
    m_w &= \sum_{n_1, \cdots, n_w}^{m_1, \cdots, m_w} n_w l_{n_1, \cdots, n_w}.
\end{align}

If a collection of $k_1, \ldots, k_d$ and $\mathbf{L}$ satisfy all of these constraints, then the value

\begin{equation*}
    s_k \binom{k}{k_1, \cdots, k_d} \binom{k_j}{\mathbf{L}} \theta_{ij}^{k_j} (a^{(j)}_{n_1, \cdots, n_w})^{l_{n_1, \cdots, n_w}}
\end{equation*}

may be pulled out to the sum of $b^{(i)}_{m_1, \cdots, m_w}$. In concise terms we have derived

{\color{red} At this point, it is infeasible to attempt to write Equation (\ref{eqn:scalar y expansion}) directly as a power series. Instead, we will extract the coefficients $b_i$ for $i = 0, 1, 2, \ldots$ by observing constraints on the coefficients.\\

First, as a result of the multinomial theorem, we have a constraint on the inner sum in Equation (\ref{eqn:scalar y expansion}), which is constraint on the index of the sum so we shall call this an index constraint.\\

Second, the coefficient $b_i$ is associated with the term $z^i$ so we must form an equality between $i$ and the power on $z$ in Equation (\ref{eqn:scalar y expansion}). To satisfy this equality, a constant is required. This is a constraint on the power of the scalar $z$ so we shall call this a power constraint. This constraint implies $k_{i + 1} = k_{i + 2} = \cdots = 0$. Therefore, both the index constraint and power constraint can be reduced to a finite series instead of an infinite series.}

\begin{equation}
    b^{(i)}_{m_1, \cdots, m_w} = \sum_{k=0}^{\infty} s_k \sum_{k_1 + \cdots + k_d = k} \binom{k}{k_1, \cdots, k_d} \prod_{j=1}^{d} \theta_{ij}^{k_j} \sum_{\substack{l_{0, \cdots, 0} + \cdots + l_{\infty, \cdots, \infty} = k_j \\ \sum_{n_1, \cdots, n_w}^{m_1, \cdots, m_w} n_1 l_{n_1, \cdots, n_w} = m_1 \\ \vdots \\ \sum_{n_1, \cdots, n_w}^{m_1, \cdots, m_w} n_w l_{n_1, \cdots, n_w} = m_w}} \binom{k_j}{\mathbf{L}} (a^{(j)}_{n_1, \cdots, n_w})^{l_{n_1, \cdots, n_w}}
\end{equation}

The only issue is that of finding solutions to the index and power constraints.

\subsubsection{Interpretation}

We can use this method for deriving power series for repeatedly composed functions with linear transformations (i.e. $\mathbf{y} = f(\mathbf{\Theta}_3 f(\mathbf{
\Theta}_1 f(\mathbf{\Theta}_1 \mathbf{x})))$). We represent our original input variable as $\mathbf{x} = \mathbf{z}$ so that we have coefficients $a_{1, 0, \cdots, 0} = a_{0, 1, \cdots, 0} = \cdots = a_{0, 0, \cdots, 1} = 1$ with all other coefficients zero. Then, we update $\mathbf{a}$ for each composition of the function to obtain $\mathbf{b}$ for $\mathbf{y}$. This can be easily applied to neural networks.\\
If a collection of $k_j$ for $j = 0, 1, 2, \ldots$ satisfy both of these constraints, then, the constant coefficients of $z^i$ may be pulled out and added to the sum of $b_i$. In concise terms we have derived that

In order to make this method computationally feasible, we must truncate the power series at some precision. Say the maximum power we wish on any variable is $K$, then, we simply replace instances of $\sum_{k=0}^{\infty} s_k x^k$ with $\sum_{k=0}^{K} s_k x^k$. Thus, coefficients of order $0, 1, \cdots, K$ will have no error and our approximation will only have truncation error from dropping terms with order $> K$.

\subsubsection{Constraint Analysis}

The constraints form a system called a restricted partition which is a concept within number theory. Fel [REF] has found an explicit solution to the constraints found in the simplified ANN case. We can construct an algorithm that operates in $\mathcal{O}(K^w)$ time by iterating through each of the indices $l_{0, \cdots, 0}, \cdots, l_{m_1, \cdots, m_w}$ from $0$ to $K$ and checking if they satisfy the constraints. This warrants some future study to see if there are ways to reduce the operation time and find solutions to the system of constraints more quickly.

%End of vector coefficients derivation.%

%If no reference, prove that the primal/dual is unique.
%Weidmann J. (1980) Linear operators and their adjoints. In: Linear Operators in Hilbert Spaces. Graduate Texts in Mathematics, vol 68. Springer, New York, NY
%Riesz representation theorem for proving that the dual is unique???


\input{results}

\section{Discussion}\label{discussion}

% Find \alpha_k for neural network (NN) expansion
% Find \beta_k for constitutive relationship (CR) expansion
% Solve for C_k through fit (least-squares for now) of \alpha_k to \beta_k

% Hadamard (element-wise) product does not distribute.
In the solution of the

% data pre-processing
To avoid implicit bias, data preprocessing is an important first step in training a neural network. Commonly, data is whitened, also known as scaling or standardizing, ${\bf x}_s: {\bf x}_s = \frac{{\bf x} - \overline{\bf x}}{\sigma}$, where $\overline{\bf x}$ is the arithmetic mean and $\sigma$ is the standard deviation of the data set, ${\bf x}$. However, a model trained on such scaled data would change the space such that the primal space of the model (scaled) would differ from the primal space of the constitutive relationship (unscaled).

To accommodate whitened data, which is necessary to remove implicit bias while training a neural network, the constitutive relationship must also operate in the scaled primal space. After model training and fit of the physical constants in the constitutive relationship from $\boldsymbol{\alpha}$, the dual space of the neural network expansion, and $\boldsymbol{\beta}$, the dual space of the constitutive relationship expansion, the inverse transformation should be applied to closed-form constitutive relationship to extract the physical constant, e.g.
\begin{align*}
    f(x) &= C_0 + C_1 x_s \\
        &= C_0 + C_1 \left( \frac{x - \overline{x}}{\sigma} \right) \\
        &= C_0^\prime + C_1^\prime x
\end{align*}
where
\begin{conditions}
    C_0^\prime & $C_0 - C_1 \overline{x}/\sigma$ \\
    C_1^\prime & $C_1/\sigma$
\end{conditions}
Note that this accommodation may introduce a bias/offset even in constitutive relationships where no such offset was present before.

% comment on the distance metric
The Euclidean ($L_2\textrm{-norm}$) distance is used in optimizations, such as the training of neural networks and other machine learning algorithms, through the mean-square error,
\begin{equation*}
    \argmin_{\boldsymbol{\theta}} \|{\bf y} - \boldsymbol{\theta}{\bf x}.
\end{equation*}
However, because of the curse of dimensionality, the $L_2\textrm{-norm}$ is not a desirable metric in high-dimensional spaces.

% original
Many non-trivial problems in materials science, and in science more broadly, are explained not through a single constitutive relationship, but through a superposition of contributing physics.

Figure~\ref{fig:stress} shows an artificial dataset constructed to replicate the impact of yield stress in a two-phase, solid-solution strengthened alloy system. Using a combination of composite theory for the contribution of flow stress, {\color{red} NAME} solid solution \cite{solid solution}, and Hall-Petch \cite{Hall-Petch} strengthening, the expected yield stress is

\begin{equation}
	\sigma_y = F_v^A \sigma_f^A + F_v^B \sigma_f^B + \sum_i C_i [x_i]^{2/3} + \sum_j k_j d_j^{-1/2} + \ldots
\end{equation}

\noindent with free parameters \\[2ex]
\begin{tabular}{l l}
	$F_v^i$	& Volume fraction of phase $i$ \\
	$[x_i]$	& Concentration of solute $i$ \\
	$d_j$	& Average grain diameter of phase $j$
\end{tabular}
\\[2ex]
\noindent and fixed parameters \\[2ex]
\begin{tabular}{l l}
	$\sigma_f^i$	& Flow stress of phase $i$ \\
	$C_i$		& Solid solution strengthening coefficient for solute species $i$ \\
	$k_j$		& Hall-Petch strengthening coefficient for phase $j$
\end{tabular}
\\[2ex]

The goal is to iteratively improve on this constitutive model one term at a time, and monitor the effect on the residuals between the predicted yield, $\hat{\sigma_y}$ and the actual yield $\sigma_y$.

Together, this leads to a seven-step process for systematically and incrementally extracting physics information from an ANN:
\begin{enumerate}
	\item Collect data--features and targets--for which relationships are expected to exist.
	\item Design and train a fully dense multi-layer perceptron network (ANN).
	\item Build a power series expansion from the architecture of this ANN, using Equations~(\ref{eqn:sigmoid zeta expansion}) and (\ref{eqn:ANN power series coefficient generating function}) to populate the coefficients using the trained weights from the neural network.
	\item Hypothesize a constitutive relationship between the feature space and the target space. \label{item:hypothesis}
	\item Recast the terms in the hypothesis function from \#\ref{item:hypothesis} as power series expansions, creating power series coefficient generating functions that are functions of the constitutive model fitting parameters. An example of this process is provided below, and a table of select power series expansions relevant to materials research are provided in Table~(\ref{table:power series expansions}). \label{item:coefficients}
	\item Perform an optimization, \emph{e.g.} least squares, fit to find the fitting parameters from \#\ref{item:coefficients}
	\item Calculate the residuals of the ANN power series expansion coefficient vector, and from this residual vector, the error in the model. If the accuracy is sufficient for the application, stop; otherwise, expand the constitutive relationship from step \#\ref{item:hypothesis} and repeat.
\end{enumerate}

\begin{table}[htp]
\caption{Examples of coefficient generating functions for functional forms commonly found in materials physics.}
\begin{center}
\begin{tabular}{c | c c c c}
	k	& %
		$C a^x$	& %
			$C x^n$	& %
				$Ce^{-\beta x}$	& %
					$C x^{-1/2}$ \\[2ex]
	\hline
	0	& %
		$1$	& %
			--	& %
				$C$	& %
					$C$ \\[2ex]
	1	& %
		$C \ln a$	& %
			--	& %
				$-\beta C$		& %
					$-\frac{1}{2}C$ \\[2ex]
	2	& %
		$\frac{(\ln a)^2}{2} C$	& %
			--	&  %
				$\frac{\beta^2}{2} C$	& %
					$\frac{3}{8}C$	\\[2ex]
	\vdots & \multicolumn{4}{c}{\vdots} \\[2ex]
	n	& %
		$\frac{(\ln a)^n}{n!} C$	& %
			$\left\{\begin{array}{c l}
				C & \text{if}\ k = n \\
				0 & \text{otherwise}
			  \end{array}\right.$	& %
				$(-1)^n\frac{\beta^n}{n!} C$	& %
					$C \prod_{i=1}^n (-1)\frac{2i - 1}{2i}$ \\[2ex]
	\hline
\end{tabular}
\end{center}
\label{tab:generating functions of common functions}
\end{table}%


%\noindent {\color{red} How does dropout affect this? It doesn't. Dropout simply sets specific $\boldsymbol\theta$ to zero, which is handled seamlessly in the previous treatment.}


\section{Conclusions}\label{conclusions}
A generalized mathematical framework for proposing a constitutive relationship and fitting the physical constants associated with that constitutive relationship to a data corpus using the generalized fitting framework provided by machine learning/artificial intelligence has been presented. The proposed method maps between an arbitrary constitutive relationship and an artificial neural network model. The resulting covector spaces (coefficients) of the series are colinear. The generating function for the constitutive relationship is in terms of physical constants while that of neural network is determined through the trained model parameters. The method of least-squares is used to fit the physical constants to the trained neural network model parameters through this colinear covector space.

A simplified yield strength model, which includes flow stress, solute concentration, and Hall-Petch strengthening, is provided as an example to demonstrate the more general form for constructing the vector space for these two models.

Neural network series expansions are constructed layer-wise, which allows this framework to handle arbitrarily complex network architectures, including drop out, whitening, and any element-wise activation function. Mathematical descriptions of the rectified linear unit (ReLU) activation commonly used in neural network hidden layers, linear activation for regression networks, and softmax activation for classification networks are derived.

A mathematical framework to create constitutive relationship series expansions is also provided. Select coefficient generating functions for functional forms commonly found in materials physics are presented in the Appendix (Table~\ref{tab:generating functions of common functions}) to serve as a starting point, but any other functions that can be described by a polynomial series may be used.

As a single framework capable of describing arbitrarily complex relationships, this approach is intended to facilitate fits between existing data and any hypothesized constitutive relationships built upon the same vector space as the trained neural network model.



The authors wish to thank Linus, Martin, Richard, ... for fruitful discussions. Support through XYZ grant ABC123456 is gratefully acknowledged.


\section{Appendix}\label{sec:appendix}

\subsection{Activation Functions}
\subsubsection{The logistic sigmoid}
\begin{equation}
	\sigma(x) = \frac{1}{1 + e^{-x}}
	\label{eqn:sigmoid}
\end{equation}
is a special case of the generating function for the Euler polynomial coefficients,
\begin{equation}
	\frac{2e^{x t}}{1 + e^t} = \sum_{n=0}^\infty E_n(x) \frac{t^n}{n!}
\end{equation}
where, for $x = 0$,
\begin{equation}
	\sigma(x) = \frac{1}{2} \sum_{n=0}^\infty E_n(0) \frac{(-1)^n}{n!}.
	\label{eqn:sigmoid Euler expansion}
\end{equation}

The Euler polynomials at $x=0$,
\begin{equation}
	E_n(0) = -2(n+1)^{-1} \left( 2^{n+1} - 1 \right) B_{n+1}
\end{equation}
where $B_n$ is the $n^\textrm{th}$ Bernoulli number. Since Bernoulli numbers of odd index, with the exception of $B_1$, are zero, $E_i(0) = 0$ for $i = 2, 4, 6, \ldots, 2n$. Therefore, the summand and limits of Equation~(\ref{eqn:sigmoid Euler expansion}) change to
\begin{equation}
	\sigma(x) = \frac{1}{2} - \frac{1}{2} \sum_{n=1}^\infty \left( \frac{E_{2n-1(0)}}{(2n-1)!} \right) x^{2n-1}.
\end{equation}

The series representation of $E_{2n-1}(x)$
\begin{equation}
	E_{2n-1}(x) = \frac{(-1)^n 4 (2n - 1)!}{\pi^{2n+1}} \sum_{k=0}^\infty \frac{\cos [(2k + 1) \pi x]}{(2k + 1)^{2n}}
\end{equation}
such that,
\begin{equation}
	E_{2n-1}(0) = \frac{(-1)^n 4 (2n - 1)!}{\pi^{2n+1}} \sum_{k=0}^\infty \frac{1}{(2k + 1)^{2n}}
\end{equation}
and therefore,
\begin{eqnarray}
	\sigma(x) & = & \frac{1}{2} - \sum_{n=1}^\infty 2 \frac{(-1)^n}{\pi^{2n}} \left( \sum_{k=0}^\infty \frac{1}{(2k+1)^{2n}} \right) x^{2n-1} \\
		& = & \frac{1}{2} - \sum_{n=1}^\infty 2 \frac{(-1)^n}{\pi^{2n}} \left( 4^{-n} \left( 4^n - 1 \right) \zeta(2n) \right) x^{2n-1} \nonumber \\
		& = & \frac{1}{2} - \sum_{n=1}^\infty \underbrace{2 \left( \frac{-1}{4\pi^2} \right)^n \left( 4^n - 1 \right) \zeta(2n)}_{a_n} x^{2n-1} \nonumber \\
		& = & \sum_{n=0}^\infty a_n x^n,\ a_n = \left\{ \begin{array}{l l}
			1/2	& n = 0 \\
			-2 \left( \frac{-1}{4\pi^2} \right)^{(n+1)/2} \left( 4^{(n+1)/2} - 1 \right)\zeta(n+1)	& n\ \text{odd} \\
			0	& n\ \text{even}
		\end{array}\right.
		\label{eqn:sigmoid zeta expansion}
\end{eqnarray}

\subsection{Coefficient Generating Functions for Common Functions}

Central to this approach is the connection ability to represent a constitutive relationship and a neural network on the same vector/covector space. This is done through a polynomial series expansion of both the neural network, covered in the text, and the constitutive relationship. Select generating functions are provided here.

\begin{sidewaystable}[htp]
\caption{Examples of coefficient generating functions for functional forms commonly found in materials physics.}
\begin{center}
\begin{tabular}{c | c c c c c c}
	k	& %
		$C a^x$	& %
			$C x^n$	& %
				$Ce^{-\beta x}$	& %
					$C x^{-1/2}$ & %
                        $C (1 + x)^\alpha$ & %
                            $C \ln(1 + x)$ \\[2ex]
	\hline
	0	& %
		$1$	& %
			--	& %
				$C$	& %
					$C$ & %
                        $C$ & %
                            $0$ \\[2ex]
	1	& %
		$C \ln a$	& %
			--	& %
				$-\beta C$		& %
					$-\frac{1}{2}C$ & %
                        $C\alpha$ & %
                            $C$ \\[2ex]
	2	& %
		$\frac{(\ln a)^2}{2} C$	& %
			--	&  %
				$\frac{\beta^2}{2} C$	& %
					$\frac{3}{8}C$	& %
                        $C\frac{\alpha (\alpha - 1)}{2!}$ & %
                            $\frac{-C}{2}$ \\[2ex]
	\vdots & \multicolumn{4}{c}{\vdots} \\[2ex]
	n	& %
		$\frac{(\ln a)^n}{n!} C$	& %
			$\left\{\begin{array}{c l}
				C & \text{if}\ k = n \\
				0 & \text{otherwise}
			  \end{array}\right.$	& %
				$(-1)^n\frac{\beta^n}{n!} C$	& %
					$C \prod_{i=1}^n (-1)\frac{2i - 1}{2i}$ & %
                        $C \frac{\prod_{i=1}^n \alpha - n + 1}{n!}$ & %
                            $C\frac{(-1)^{n+1}}{n}$ \\[2ex]
	\hline
    Constraint & %
        & %
            & %
                & %
                    & %
                        $-1 < x < 1$ & %
                            $-1 < x \le 1$ \\
    \hline
\end{tabular}
\end{center}
\label{tab:generating functions of common functions}
\end{sidewaystable}

%Beginning of scalar coefficients worked examples.%
\subsection{Coefficients of the Scalar Polynomial Series}

A result of the coefficient derivation of polynomial series is that we may construct formulae for the coefficients of an arbitrary function by simply solving for the constraints in Equation (\ref{eqn:scalar power series result}). We provide a few instances of worked coefficients here. 

We find $b_0$. $i = 0 \implies k_1 = k_2 = \cdots = 0$ so $k_0 = k$. Therefore,

\begin{align*}
    b_0 
    &= \sum_{k=0}^{\infty} s_k \theta^k \sum_{\substack{k_0 + k_1 + k_2 + \cdots = k \\ 0 k_0 + 1 k_1 + 2 k_2 + \cdots = 0}} \binom{k}{k_0, k_1, k_2, \cdots} \prod_{n=0}^{\infty} a_n^{k_n} \\
    &= \sum_{k=0}^{\infty} s_k \theta^k \binom{k}{k} a_0^{k} \\
    &= \sum_{k=0}^{\infty} s_k \theta^k a_0^k
\end{align*}

We find $b_1$. $i = 1 \implies k_2 = k_3 = \cdots = 0$ so $k_0 = k - 1$ and $k_1 = 1$. Therefore,

\begin{align*}
    b_1
    &= \sum_{k=0}^{\infty} s_k \theta^k \sum_{\substack{k_0 + k_1 + k_2 + \cdots = k \\ 0 k_0 + 1 k_1 + 2 k_2 + \cdots = 1}} \binom{k}{k_0, k_1, k_2, \cdots} \prod_{n=0}^{\infty} a_n^{k_n} \\
    &= \sum_{k=0}^{\infty} s_k \theta^k \binom{k}{k - 1, 1} a_0^{k - 1} a_1 \\
    &= \sum_{k=0}^{\infty} k s_k \theta^k a_0^{k-1} a_1
\end{align*}

We find $b_2$. For this coefficient, there are multiple solutions to the constraints. 

\begin{itemize}
    \item $i = 2 \implies k_3 = k_4 = k_5 = \cdots = 0$ and either
    \begin{itemize}
        \item $k_2 = 0, k_1 = 2, k_0 = k - 2$ or
        \item $k_2 = 1, k_1 = 0, k_0 = k - 1$
    \end{itemize}
\end{itemize}

Therefore,

\begin{align*}
    b_2
    &= \sum_{k=0}^{\infty} s_k \theta^k \sum_{\substack{k_0 + k_1 + k_2 + \cdots = k \\ 0 k_0 + 1 k_1 + 2 k_2 + \cdots = 2}} \binom{k}{k_0, k_1, k_2, \cdots} \prod_{n=0}^{\infty} a_n^{k_n} \\
    &= \sum_{k=0}^{\infty} s_k \theta^k \left(\binom{k}{k - 2, 2, 0}a_0^{k-2} a_1^{2} + \binom{k}{k - 1, 0, 1}a_0^{k-1} a_2\right) \\
    &= \sum_{k=0}^{\infty} s_k \theta^k \left(\frac{k(k-1)}{2} a_0^{k-2}a_1^{2} + k a_0^{k-1}a_2\right)
\end{align*}
%End of scalar coefficients worked examples.%

%\bibliographystyle{ieeetr}
%\bibliography{references}

\end{document}
