\section{Appendix}\label{sec:appendix}

\subsection{Activation Functions}
\subsubsection{The logistic sigmoid}
\begin{equation}
	\sigma(x) = \frac{1}{1 + e^{-x}}
	\label{eqn:sigmoid}
\end{equation}
is a special case of the generating function for the Euler polynomial coefficients,
\begin{equation}
	\frac{2e^{x t}}{1 + e^t} = \sum_{n=0}^\infty E_n(x) \frac{t^n}{n!}
\end{equation}
where, for $x = 0$,
\begin{equation}
	\sigma(x) = \frac{1}{2} \sum_{n=0}^\infty E_n(0) \frac{(-1)^n}{n!}.
	\label{eqn:sigmoid Euler expansion}
\end{equation}

The Euler polynomials at $x=0$,
\begin{equation}
	E_n(0) = -2(n+1)^{-1} \left( 2^{n+1} - 1 \right) B_{n+1}
\end{equation}
where $B_n$ is the $n^\textrm{th}$ Bernoulli number. Since Bernoulli numbers of odd index, with the exception of $B_1$, are zero, $E_i(0) = 0$ for $i = 2, 4, 6, \ldots, 2n$. Therefore, the summand and limits of Equation~(\ref{eqn:sigmoid Euler expansion}) change to
\begin{equation}
	\sigma(x) = \frac{1}{2} - \frac{1}{2} \sum_{n=1}^\infty \left( \frac{E_{2n-1(0)}}{(2n-1)!} \right) x^{2n-1}.
\end{equation}

The series representation of $E_{2n-1}(x)$
\begin{equation}
	E_{2n-1}(x) = \frac{(-1)^n 4 (2n - 1)!}{\pi^{2n+1}} \sum_{k=0}^\infty \frac{\cos [(2k + 1) \pi x]}{(2k + 1)^{2n}}
\end{equation}
such that,
\begin{equation}
	E_{2n-1}(0) = \frac{(-1)^n 4 (2n - 1)!}{\pi^{2n+1}} \sum_{k=0}^\infty \frac{1}{(2k + 1)^{2n}}
\end{equation}
and therefore,
\begin{eqnarray}
	\sigma(x) & = & \frac{1}{2} - \sum_{n=1}^\infty 2 \frac{(-1)^n}{\pi^{2n}} \left( \sum_{k=0}^\infty \frac{1}{(2k+1)^{2n}} \right) x^{2n-1} \\
		& = & \frac{1}{2} - \sum_{n=1}^\infty 2 \frac{(-1)^n}{\pi^{2n}} \left( 4^{-n} \left( 4^n - 1 \right) \zeta(2n) \right) x^{2n-1} \nonumber \\
		& = & \frac{1}{2} - \sum_{n=1}^\infty \underbrace{2 \left( \frac{-1}{4\pi^2} \right)^n \left( 4^n - 1 \right) \zeta(2n)}_{a_n} x^{2n-1} \nonumber \\
		& = & \sum_{n=0}^\infty a_n x^n,\ a_n = \left\{ \begin{array}{l l}
			1/2	& n = 0 \\
			-2 \left( \frac{-1}{4\pi^2} \right)^{(n+1)/2} \left( 4^{(n+1)/2} - 1 \right)\zeta(n+1)	& n\ \text{odd} \\
			0	& n\ \text{even}
		\end{array}\right.
		\label{eqn:sigmoid zeta expansion}
\end{eqnarray}

\subsection{Coefficient Generating Functions for Common Functions}

Central to this approach is the connection ability to represent a constitutive relationship and a neural network on the same vector/covector space. This is done through a polynomial series expansion of both the neural network, covered in the text, and the constitutive relationship. Select generating functions are provided here.

\begin{sidewaystable}[htp]
\caption{Examples of coefficient generating functions for functional forms commonly found in materials physics.}
\begin{center}
\begin{tabular}{c | c c c c c c}
	k	& %
		$C a^x$	& %
			$C x^n$	& %
				$Ce^{-\beta x}$	& %
					$C x^{-1/2}$ & %
                        $C (1 + x)^\alpha$ & %
                            $C \ln(1 + x)$ \\[2ex]
	\hline
	0	& %
		$1$	& %
			--	& %
				$C$	& %
					$C$ & %
                        $C$ & %
                            $0$ \\[2ex]
	1	& %
		$C \ln a$	& %
			--	& %
				$-\beta C$		& %
					$-\frac{1}{2}C$ & %
                        $C\alpha$ & %
                            $C$ \\[2ex]
	2	& %
		$\frac{(\ln a)^2}{2} C$	& %
			--	&  %
				$\frac{\beta^2}{2} C$	& %
					$\frac{3}{8}C$	& %
                        $C\frac{\alpha (\alpha - 1)}{2!}$ & %
                            $\frac{-C}{2}$ \\[2ex]
	\vdots & \multicolumn{4}{c}{\vdots} \\[2ex]
	n	& %
		$\frac{(\ln a)^n}{n!} C$	& %
			$\left\{\begin{array}{c l}
				C & \text{if}\ k = n \\
				0 & \text{otherwise}
			  \end{array}\right.$	& %
				$(-1)^n\frac{\beta^n}{n!} C$	& %
					$C \prod_{i=1}^n (-1)\frac{2i - 1}{2i}$ & %
                        $C \frac{\prod_{i=1}^n \alpha - n + 1}{n!}$ & %
                            $C\frac{(-1)^{n+1}}{n}$ \\[2ex]
	\hline
    Constraint & %
        & %
            & %
                & %
                    & %
                        $-1 < x < 1$ & %
                            $-1 < x \le 1$ \\
    \hline
\end{tabular}
\end{center}
\label{tab:generating functions of common functions}
\end{sidewaystable}
